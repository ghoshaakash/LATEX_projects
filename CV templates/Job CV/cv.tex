\documentclass{resume} % Use the custom resume.cls style
\usepackage{amsmath}
\usepackage{amssymb}
\usepackage[left=0.4 in,top=0.4in,right=0.4 in,bottom=0.4in]{geometry} % Document margins
\newcommand{\tab}[1]{\hspace{.2667\textwidth}\rlap{#1}} 
\newcommand{\itab}[1]{\hspace{0em}\rlap{#1}}
\name{Aakash Ghosh} % Your name
% You can merge both of these into a single line, if you do not have a website.
\address{+91 6296708277\\ IISER Kolkata} 
\address{\href{mailto:ag19ms12@iiserkol.ac.in}{ag19ms12@iiserkol.ac.in} \\  \href{https://ghoshaakash.pages.dev/}{ghoshaakash.pages.dev}}  %

\begin{document}

%----------------------------------------------------------------------------------------
%	OBJECTIVE
%----------------------------------------------------------------------------------------

\begin{rSection}{Background}

I am a fifth year undergrad in the department of mathematics and statistics (DMS) at IISER Kolkata.\\
IISER is designed
to bring out the motivated young minds in the nation towards scientific research right from their undergraduate years.
As I have progressed through my undergraduate education, I have become more invested in pursuing a fast-paced career around data and finance.

\end{rSection}
%----------------------------------------------------------------------------------------
%	EDUCATION SECTION
%----------------------------------------------------------------------------------------

\begin{rSection}{Education}

{\bf 5 Year Bachelor of Science and Master of Science (BS-MS) Dual Degree Programme}\\ IISER Kolkata \hfill {Expected 2024}\\
Relevant Coursework: Machine Learning, Statistics(2 courses), Probability(2 courses), Data structures(in C), Introductory Economics and Econometrics(in R)\\
I am currently studying combinatorial game theory for my thesis and working on deep learning for an independent study
\end{rSection}



%----------------------------------------------------------------------------------------
% TECHINICAL STRENGTHS	
%----------------------------------------------------------------------------------------
\begin{rSection}{SKILLS}
I am proficient in C/C++, Python and R. In python, I have worked with NumPy, pandas, scikit-learn and TensorFlow.  

\end{rSection}

\begin{rSection}{previous EXPERIENCE}

\textbf{Some results on the exponential of a matrix} \hfill 2021\\
Advisor: \textbf{Dr. Somnath Basu} \hfill \textit{IISER Kolkata}
 \begin{itemize}
    \itemsep -3pt {} 
     \item Did readings on analysis and topology.
     \item Showed $GL_n(\mathbb{R})$ forms a metric space under operator norm. This allows us to show convergence of the series $\sum_{n\in\mathbb{N}}A^n/n!$ and therefore conclude $e^A$ is well defined. I also proved a few related result related to it's determinant. \href{https://ghoshaakash.pages.dev/assets/Exponent_of_a_matrix(5).pdf}{\textbf{Link to write-up}}
 \end{itemize}
 
\textbf{Number theory and Cryptogrphy} \hfill 2020\\
Advisor: \textbf{Dr. Avishek Adhikari} \hfill \textit{Presidency University}
 \begin{itemize}
    \itemsep -3pt {} 
     \item  Read about elementary number theory and the required algebra
     \item Read about basics of cryptography
     \item Read about visual cryptography and implemented them in python and C++
     \item Read about secret sharing using Chinese remainder theorem \href{https://1drv.ms/b/s!AqlhppskyZ0YgYEgR_u7LGbqv2wuUQ?e=rezStX}{\textbf{Link to presentation}}
 \end{itemize}
\end{rSection} 

%--------algebraic graph theory(self)--------------------------------------------------------------------------------
%	WORK EXPERIENCE SECTION
%----------------------------------------------------------------------------------------

    
\end{rSection}
\begin{rSection}{Teaching EXPERIENCE} 
        \begin{itemize}
            \item 	I am a Teaching Assistant for Data Structures and Algorithms for Autumn semester,2023
        \end{itemize}       
\end{rSection}
%----------------------------------------------------------------------------------------
\begin{rSection}{Extra-Curricular Activities} 
\begin{itemize}
    \item 	I am very interested in solving puzzles and riddles
    \item	I am quite interested in Chess and other games. Recently, I have been looking for efficient ways of playing them.
\end{itemize}


\end{rSection}


\end{document}
