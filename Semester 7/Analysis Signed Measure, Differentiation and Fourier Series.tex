\documentclass{tufte-book}

\hypersetup{colorlinks}% uncomment this line if you prefer colored hyperlinks (e.g., for onscreen viewing)

%%
% Book metadata





\title{Analysis-5}

\author[Aakash Ghosh]{Aakash Ghosh (19MS129)}







%%
% If they're installed, use Bergamo and Chantilly from www.fontsite.com.
% They're clones of Bembo and Gill Sans, respectively.
%\IfFileExists{bergamo.sty}{\usepackage[osf]{bergamo}}{}% Bembo
%\IfFileExists{chantill.sty}{\usepackage{chantill}}{}% Gill Sans

%\usepackage{microtype}

%%
% Just some sample text
\usepackage{lipsum}

%%
% For nicely typeset tabular material
\usepackage{booktabs}

%%
% For graphics / images
\usepackage{graphicx}
\setkeys{Gin}{width=\linewidth,totalheight=\textheight,keepaspectratio}
\graphicspath{{graphics/}}

% The fancyvrb package lets us customize the formatting of verbatim
% environments.  We use a slightly smaller font.
\usepackage{fancyvrb}
\fvset{fontsize=\normalsize}

%%
% Prints argument within hanging parentheses (i.e., parentheses that take
% up no horizontal space).  Useful in tabular environments.
\newcommand{\hangp}[1]{\makebox[0pt][r]{(}#1\makebox[0pt][l]{)}}

%%
% Prints an asterisk that takes up no horizontal space.
% Useful in tabular environments.
\newcommand{\hangstar}{\makebox[0pt][l]{*}}

%%
% Prints a trailing space in a smart way.
\usepackage{xspace}

%%
% Some shortcuts for Tufte's book titles.  The lowercase commands will
% produce the initials of the book title in italics.  The all-caps commands
% will print out the full title of the book in italics.
\newcommand{\vdqi}{\textit{VDQI}\xspace}
\newcommand{\ei}{\textit{EI}\xspace}
\newcommand{\ve}{\textit{VE}\xspace}
\newcommand{\be}{\textit{BE}\xspace}
\newcommand{\VDQI}{\textit{The Visual Display of Quantitative Information}\xspace}
\newcommand{\EI}{\textit{Envisioning Information}\xspace}
\newcommand{\VE}{\textit{Visual Explanations}\xspace}
\newcommand{\BE}{\textit{Beautiful Evidence}\xspace}

\newcommand{\TL}{Tufte-\LaTeX\xspace}

% Prints the month name (e.g., January) and the year (e.g., 2008)
\newcommand{\monthyear}{%
	\ifcase\month\or January\or February\or March\or April\or May\or June\or
	July\or August\or September\or October\or November\or
	December\fi\space\number\year
}


% Prints an epigraph and speaker in sans serif, all-caps type.
\newcommand{\openepigraph}[2]{%
	%\sffamily\fontsize{14}{16}\selectfont
	\begin{fullwidth}
	\sffamily\large
	\begin{doublespace}
	\noindent\allcaps{#1}\\% epigraph
	\noindent\allcaps{#2}% author
	\end{doublespace}
	\end{fullwidth}
}

% Inserts a blank page
\newcommand{\blankpage}{\newpage\hbox{}\thispagestyle{empty}\newpage}

\usepackage{units}

% Typesets the font size, leading, and measure in the form of 10/12x26 pc.
\newcommand{\measure}[3]{#1/#2$\times$\unit[#3]{pc}}

% Macros for typesetting the documentation
\newcommand{\hlred}[1]{\textcolor{Maroon}{#1}}% prints in red
\newcommand{\hangleft}[1]{\makebox[0pt][r]{#1}}
\newcommand{\hairsp}{\hspace{1pt}}% hair space
\newcommand{\hquad}{\hskip0.5em\relax}% half quad space
\newcommand{\TODO}{\textcolor{red}{\bf TODO!}\xspace}
\newcommand{\na}{\quad--}% used in tables for N/A cells
\providecommand{\XeLaTeX}{X\lower.5ex\hbox{\kern-0.15em\reflectbox{E}}\kern-0.1em\LaTeX}
\newcommand{\tXeLaTeX}{\XeLaTeX\index{XeLaTeX@\protect\XeLaTeX}}
% \index{\texttt{\textbackslash xyz}@\hangleft{\texttt{\textbackslash}}\texttt{xyz}}
\newcommand{\tuftebs}{\symbol{'134}}% a backslash in tt type in OT1/T1
\newcommand{\doccmdnoindex}[2][]{\texttt{\tuftebs#2}}% command name -- adds backslash automatically (and doesn't add cmd to the index)
\newcommand{\doccmddef}[2][]{%
	\hlred{\texttt{\tuftebs#2}}\label{cmd:#2}%
	\ifthenelse{\isempty{#1}}%
		{% add the command to the index
			\index{#2 command@\protect\hangleft{\texttt{\tuftebs}}\texttt{#2}}% command name
		}%
		{% add the command and package to the index
			\index{#2 command@\protect\hangleft{\texttt{\tuftebs}}\texttt{#2} (\texttt{#1} package)}% command name
			\index{#1 package@\texttt{#1} package}\index{packages!#1@\texttt{#1}}% package name
		}%
}% command name -- adds backslash automatically
\newcommand{\doccmd}[2][]{%
	\texttt{\tuftebs#2}%
	\ifthenelse{\isempty{#1}}%
		{% add the command to the index
			\index{#2 command@\protect\hangleft{\texttt{\tuftebs}}\texttt{#2}}% command name
		}%
		{% add the command and package to the index
			\index{#2 command@\protect\hangleft{\texttt{\tuftebs}}\texttt{#2} (\texttt{#1} package)}% command name
			\index{#1 package@\texttt{#1} package}\index{packages!#1@\texttt{#1}}% package name
		}%
}% command name -- adds backslash automatically
\newcommand{\docopt}[1]{\ensuremath{\langle}\textrm{\textit{#1}}\ensuremath{\rangle}}% optional command argument
\newcommand{\docarg}[1]{\textrm{\textit{#1}}}% (required) command argument
\newenvironment{docspec}{\begin{quotation}\ttfamily\parskip0pt\parindent0pt\ignorespaces}{\end{quotation}}% command specification environment
\newcommand{\docenv}[1]{\texttt{#1}\index{#1 environment@\texttt{#1} environment}\index{environments!#1@\texttt{#1}}}% environment name
\newcommand{\docenvdef}[1]{\hlred{\texttt{#1}}\label{env:#1}\index{#1 environment@\texttt{#1} environment}\index{environments!#1@\texttt{#1}}}% environment name
\newcommand{\docpkg}[1]{\texttt{#1}\index{#1 package@\texttt{#1} package}\index{packages!#1@\texttt{#1}}}% package name
\newcommand{\doccls}[1]{\texttt{#1}}% document class name
\newcommand{\docclsopt}[1]{\texttt{#1}\index{#1 class option@\texttt{#1} class option}\index{class options!#1@\texttt{#1}}}% document class option name
\newcommand{\docclsoptdef}[1]{\hlred{\texttt{#1}}\label{clsopt:#1}\index{#1 class option@\texttt{#1} class option}\index{class options!#1@\texttt{#1}}}% document class option name defined
\newcommand{\docmsg}[2]{\bigskip\begin{fullwidth}\noindent\ttfamily#1\end{fullwidth}\medskip\par\noindent#2}
\newcommand{\docfilehook}[2]{\texttt{#1}\index{file hooks!#2}\index{#1@\texttt{#1}}}
\newcommand{\doccounter}[1]{\texttt{#1}\index{#1 counter@\texttt{#1} counter}}

% Generates the index
\usepackage{makeidx}
\makeindex


\usepackage{amsmath}
\usepackage{amsthm}

\newtheorem{defn}{Definition}
\newtheorem{theorem}{Theorem}
\newtheorem{corollary}{Corollary}[theorem]
\newtheorem{lemma}[theorem]{Lemma}



% chapter format
\titleformat{\chapter}%
  {\huge\rmfamily\itshape\color[HTML]{1B262C}}% format applied to label+text
  {\llap{\colorbox[HTML]{1B262C}{\parbox{1.5cm}{\hfill\itshape\huge\color{white}\thechapter}}}}% label
  {2pt}% horizontal separation between label and title body
  {}% before the title body
  []% after the title body

% section format
\titleformat{\section}%
  {\normalfont\Large\itshape\color[HTML]{0F4C75}}% format applied to label+text
  {\llap{\colorbox[HTML]{0F4C75}{\parbox{1.5cm}{\hfill\color{white}\thesection}}}}% label
  {1em}% horizontal separation between label and title body
  {}% before the title body
  []% after the title body

% subsection format
\titleformat{\subsection}%
  {\normalfont\large\itshape\color[HTML]{3282B8}}% format applied to label+text
  {\llap{\colorbox[HTML]{3282B8}{\parbox{1.5cm}{\hfill\color{white}\thesubsection}}}}% label
  {1em}% horizontal separation between label and title body
  {}% before the title body
  []% after the title body

  \setcounter{secnumdepth}{1}

\usepackage{tcolorbox}


\begin{document}



\maketitle




% r.5 contents
\tableofcontents



\chapter{Signed Measure}
\section{Introduction}
\begin{defn}[Signed Measure]
	Given a measurable space $(X,\mathcal{M})$, a signed measure is a function $\nu:\mathcal{M}\to[-\infty,\infty]$ with the following properties:
\begin{enumerate}
	\item $\nu(\phi)=0$
	\item $\nu$ can assume either $\infty$ or $-\infty$ but not both 
	\item If $\{E_j\}$ is a sequence of disjoint sets in $\mathcal{M}$, then $\sum_{i=1}^\infty \nu(E_i)=\nu\left(\cup_{i=1}^\infty E_i \right)$
\end{enumerate}
\end{defn}
\marginnote{	
One should note that normal measures are also signed measure, the only difference is the extension of the range of the measure function to cover almost all of $\mathbb R$.
}
\section{Upper and lower continuity}

\begin{theorem}[Uppercontinuity]
	Let $\{E_i\}$ be a countable collection of measurable set with $E_i\subseteq E_{i+1}$. Then:
	\begin{align}
		\lim_{i\to\infty}\nu(E_i)=\nu\left(\bigcup_{i=1}^\infty E_i\right)
	\end{align}
\end{theorem}

\begin{theorem}[Lowercontinuity]
	Let $\{E_i\}$ be a countable collection of measurable set with $E_{i+1}\subseteq E_{i}$. Then:
	\begin{align}
		\lim_{i\to\infty}\nu(E_i)=\nu\left(\bigcap_{i=1}^\infty E_i\right)
	\end{align}
\end{theorem}

\begin{proof}
Same as what we do for unsigned measure
\end{proof}

\section{Positive,Negetive and Null Set}

\begin{defn}[Positive set]
	A set whose every mesurable subset $E$ satisfies $\nu(E)\geq0$ is called a positive set.
\end{defn}
\noindent In a similar fashion we define :

\begin{defn}[Negetive set]
	A set whose every mesurable subset $E$ satisfies $\nu(E)\leq0$ is called a positive set.
\end{defn}

\begin{defn}[Null set]
	A set whose every mesurable subset $E$ satisfies $\nu(E)=0$ is called a positive set.
\end{defn}

\noindent We consider an example. Let $\mu$ be an unsigned measure and let $f$ be a measurable $L^1$ function . Let us define a measure $nu$ as:
\begin{align}
	\nu(E)=\int_Efd\mu
\end{align}
Then $\nu$ is a signed measure. If $E$ is a set such that $f\geq0$ $\mu.a.e$ on $E$ then $E$ is a positive set. Similarly we can find negetive and null sets.

\begin{lemma}
	\begin{enumerate}
		\item Subsets of positive sets are positive
		\item Countable\sidenote{A countable union is needed as in case of uncountable union, there will be a chance that the union will not belong to the sigma algebra; a sigma algebra is closed in countable union and not under arbitary union} union of positive sets are positive
	\end{enumerate}
	Similar results are also valid for null and negetive sets.
\end{lemma}

\noindent The next lemma will be required for the proof of \textbf{Hahn Decomposition Theorem} in the next section.


\begin{lemma}
	Let $\nu$ be a signed measure which doesn't attain $\infty$. A set with a positive measure has a positive subset.
\end{lemma}

\section{Hanh Decomposition}

\begin{theorem}[Hanh Decompositiopn Theorem]
	lf $\nu$ is a signed measure on $(X,\mathcal M)$, there 
	exist a positive set $P$ and a negative set $N$ for $\nu$ such that $P\cup N = X$ and $P\cap N =\phi$ 
	Moreover if  $P', N'$ is another such pair, then $P\Delta P' (= N \Delta N')$ is null in $\nu$.
\end{theorem}
\noindent\textbf{\textit{Proof Outline :}}
\begin{enumerate}
	\item Define $m=\sup_{\text{positive sets}}\nu(P)$
	\item Take a sequence $\{p_i\}$ such that $\lim_{i\to\infty}\nu(p_i)=m$
	\item Show if $P=\bigcup p_i$ then $\nu(P)=m$
	\item Show if $N=P^c$ and if $N$ has a set with positive measure, then by lemma 4, there is contradiction.
	\item If $E\subseteq P\Delta P'$ and $\nu(E)\ne0$. Without loss of generality assume $E\subseteq P$. Then $E\subseteq P'^c=N'$ which contradicts negetivity of $N'$ 
\end{enumerate}



\backmatter

\bibliography{sample-handout}
\bibliographystyle{plainnat}


\printindex

\end{document}

