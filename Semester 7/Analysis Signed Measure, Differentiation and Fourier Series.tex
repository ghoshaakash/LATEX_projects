\documentclass[notoc]{tufte-book}

\hypersetup{colorlinks}% uncomment this line if you prefer colored hyperlinks (e.g., for onscreen viewing)

%%
% Book metadata





\title{{Analysis 5} }

\author[Aakash Ghosh]{Aakash Ghosh (19MS129)}







%%
% If they're installed, use Bergamo and Chantilly from www.fontsite.com.
% They're clones of Bembo and Gill Sans, respectively.
%\IfFileExists{bergamo.sty}{\usepackage[osf]{bergamo}}{}% Bembo
%\IfFileExists{chantill.sty}{\usepackage{chantill}}{}% Gill Sans

%\usepackage{microtype}

%%
% Just some sample text
\usepackage{lipsum}

%%
% For nicely typeset tabular material
\usepackage{booktabs}

%%
% For graphics / images
\usepackage{graphicx}
\setkeys{Gin}{width=\linewidth,totalheight=\textheight,keepaspectratio}
\graphicspath{{graphics/}}

% The fancyvrb package lets us customize the formatting of verbatim
% environments.  We use a slightly smaller font.
\usepackage{fancyvrb}
\fvset{fontsize=\normalsize}

%%
% Prints argument within hanging parentheses (i.e., parentheses that take
% up no horizontal space).  Useful in tabular environments.
\newcommand{\hangp}[1]{\makebox[0pt][r]{(}#1\makebox[0pt][l]{)}}

%%
% Prints an asterisk that takes up no horizontal space.
% Useful in tabular environments.
\newcommand{\hangstar}{\makebox[0pt][l]{*}}

%%
% Prints a trailing space in a smart way.
\usepackage{xspace}

%%
% Some shortcuts for Tufte's book titles.  The lowercase commands will
% produce the initials of the book title in italics.  The all-caps commands
% will print out the full title of the book in italics.
\newcommand{\vdqi}{\textit{VDQI}\xspace}
\newcommand{\ei}{\textit{EI}\xspace}
\newcommand{\ve}{\textit{VE}\xspace}
\newcommand{\be}{\textit{BE}\xspace}
\newcommand{\VDQI}{\textit{The Visual Display of Quantitative Information}\xspace}
\newcommand{\EI}{\textit{Envisioning Information}\xspace}
\newcommand{\VE}{\textit{Visual Explanations}\xspace}
\newcommand{\BE}{\textit{Beautiful Evidence}\xspace}

\newcommand{\TL}{Tufte-\LaTeX\xspace}

% Prints the month name (e.g., January) and the year (e.g., 2008)
\newcommand{\monthyear}{%
	\ifcase\month\or January\or February\or March\or April\or May\or June\or
	July\or August\or September\or October\or November\or
	December\fi\space\number\year
}


% Prints an epigraph and speaker in sans serif, all-caps type.
\newcommand{\openepigraph}[2]{%
	%\sffamily\fontsize{14}{16}\selectfont
	\begin{fullwidth}
	\sffamily\large
	\begin{doublespace}
	\noindent\allcaps{#1}\\% epigraph
	\noindent\allcaps{#2}% author
	\end{doublespace}
	\end{fullwidth}
}

% Inserts a blank page
\newcommand{\blankpage}{\newpage\hbox{}\thispagestyle{empty}\newpage}

\usepackage{units}

% Typesets the font size, leading, and measure in the form of 10/12x26 pc.
\newcommand{\measure}[3]{#1/#2$\times$\unit[#3]{pc}}

% Macros for typesetting the documentation
\newcommand{\hlred}[1]{\textcolor{Maroon}{#1}}% prints in red
\newcommand{\hangleft}[1]{\makebox[0pt][r]{#1}}
\newcommand{\hairsp}{\hspace{1pt}}% hair space
\newcommand{\hquad}{\hskip0.5em\relax}% half quad space
\newcommand{\TODO}{\textcolor{red}{\bf TODO!}\xspace}
\newcommand{\na}{\quad--}% used in tables for N/A cells
\providecommand{\XeLaTeX}{X\lower.5ex\hbox{\kern-0.15em\reflectbox{E}}\kern-0.1em\LaTeX}
\newcommand{\tXeLaTeX}{\XeLaTeX\index{XeLaTeX@\protect\XeLaTeX}}
% \index{\texttt{\textbackslash xyz}@\hangleft{\texttt{\textbackslash}}\texttt{xyz}}
\newcommand{\tuftebs}{\symbol{'134}}% a backslash in tt type in OT1/T1
\newcommand{\doccmdnoindex}[2][]{\texttt{\tuftebs#2}}% command name -- adds backslash automatically (and doesn't add cmd to the index)
\newcommand{\doccmddef}[2][]{%
	\hlred{\texttt{\tuftebs#2}}\label{cmd:#2}%
	\ifthenelse{\isempty{#1}}%
		{% add the command to the index
			\index{#2 command@\protect\hangleft{\texttt{\tuftebs}}\texttt{#2}}% command name
		}%
		{% add the command and package to the index
			\index{#2 command@\protect\hangleft{\texttt{\tuftebs}}\texttt{#2} (\texttt{#1} package)}% command name
			\index{#1 package@\texttt{#1} package}\index{packages!#1@\texttt{#1}}% package name
		}%
}% command name -- adds backslash automatically
\newcommand{\doccmd}[2][]{%
	\texttt{\tuftebs#2}%
	\ifthenelse{\isempty{#1}}%
		{% add the command to the index
			\index{#2 command@\protect\hangleft{\texttt{\tuftebs}}\texttt{#2}}% command name
		}%
		{% add the command and package to the index
			\index{#2 command@\protect\hangleft{\texttt{\tuftebs}}\texttt{#2} (\texttt{#1} package)}% command name
			\index{#1 package@\texttt{#1} package}\index{packages!#1@\texttt{#1}}% package name
		}%
}% command name -- adds backslash automatically
\newcommand{\docopt}[1]{\ensuremath{\langle}\textrm{\textit{#1}}\ensuremath{\rangle}}% optional command argument
\newcommand{\docarg}[1]{\textrm{\textit{#1}}}% (required) command argument
\newenvironment{docspec}{\begin{quotation}\ttfamily\parskip0pt\parindent0pt\ignorespaces}{\end{quotation}}% command specification environment
\newcommand{\docenv}[1]{\texttt{#1}\index{#1 environment@\texttt{#1} environment}\index{environments!#1@\texttt{#1}}}% environment name
\newcommand{\docenvdef}[1]{\hlred{\texttt{#1}}\label{env:#1}\index{#1 environment@\texttt{#1} environment}\index{environments!#1@\texttt{#1}}}% environment name
\newcommand{\docpkg}[1]{\texttt{#1}\index{#1 package@\texttt{#1} package}\index{packages!#1@\texttt{#1}}}% package name
\newcommand{\doccls}[1]{\texttt{#1}}% document class name
\newcommand{\docclsopt}[1]{\texttt{#1}\index{#1 class option@\texttt{#1} class option}\index{class options!#1@\texttt{#1}}}% document class option name
\newcommand{\docclsoptdef}[1]{\hlred{\texttt{#1}}\label{clsopt:#1}\index{#1 class option@\texttt{#1} class option}\index{class options!#1@\texttt{#1}}}% document class option name defined
\newcommand{\docmsg}[2]{\bigskip\begin{fullwidth}\noindent\ttfamily#1\end{fullwidth}\medskip\par\noindent#2}
\newcommand{\docfilehook}[2]{\texttt{#1}\index{file hooks!#2}\index{#1@\texttt{#1}}}
\newcommand{\doccounter}[1]{\texttt{#1}\index{#1 counter@\texttt{#1} counter}}

% Generates the index
\usepackage{makeidx}
\makeindex


\usepackage{amsmath}




\definecolor{c1}{HTML}{000000}
\definecolor{c2}{HTML}{1E5128}
\definecolor{c5}{HTML}{4E9F3D}


\definecolor{c3}{HTML}{4E9F3D}
\definecolor{c4}{HTML}{F2FFE9}

% chapter format
\titleformat{\chapter}%
  {\huge\rmfamily\itshape\color{c1}}% format applied to label+text
  {\llap{\colorbox{c1}{\parbox{1.5cm}{\hfill\itshape\huge\color{white}\thechapter}}}}% label
  {2pt}% horizontal separation between label and title body
  {}% before the title body
  []% after the title body

% section format
\titleformat{\section}%
  {\normalfont\Large\itshape\color{c2}}% format applied to label+text
  {\llap{\colorbox{c2}{\parbox{1.5cm}{\hfill\color{white}\thesection}}}}% label
  {1em}% horizontal separation between label and title body
  {}% before the title body
  []% after the title body

% subsection format
\titleformat{\subsection}%
  {\normalfont\large\itshape\color{c5}}% format applied to label+text
  {\llap{\colorbox{c5}{\parbox{1.5cm}{\hfill\color{white}\thesubsection}}}}% label
  {1em}% horizontal separation between label and title body
  {}% before the title body
  []% after the title body

  \setcounter{secnumdepth}{2}

\let\orupee\rupee
\def\rupee{\ifmmode\text{\orupee}\else\orupee\fi}

\usepackage{tcolorbox}
\usepackage{amsthm}
\usepackage{listings}
\usepackage{color}
\usepackage{tikz,lipsum,lmodern}
\definecolor{dkgreen}{rgb}{0,0.6,0}
\definecolor{gray}{rgb}{0.5,0.5,0.5}
\definecolor{mauve}{rgb}{0.58,0,0.82}


\usepackage{physics}
\usepackage{amsmath}
\usepackage{tikz}
\usepackage{mathdots}
\usepackage{yhmath}
\usepackage{cancel}
\usepackage{color}
\usepackage{array}
\usepackage{multirow}
\usepackage{amssymb}
\usepackage{gensymb}
\usepackage{tabularx}
\usepackage{extarrows}
\usepackage{booktabs}
\usetikzlibrary{fadings}
\usetikzlibrary{patterns}
\usetikzlibrary{shadows.blur}
\usetikzlibrary{shapes}

\setcounter{tocdepth}{2}


\usepackage{hyperref}
\hypersetup{
    colorlinks=true,
    linkcolor=c2
    }

\newtheorem{defn}{Definition}
\newtheorem{theorem}{Theorem}
\newtheorem{corollary}{Corollary}[theorem]
\newtheorem{lemma}[theorem]{Lemma}
\newtheorem{remark}{Remark}


\makeatletter
% Paragraph indentation and separation for normal text
\renewcommand{\@tufte@reset@par}{%
  \setlength{\RaggedRightParindent}{0.0pc}%
  \setlength{\JustifyingParindent}{0.0pc}%
  \setlength{\parindent}{0pc}%
  \setlength{\parskip}{0pt}%
}
\@tufte@reset@par
\makeatother









\begin{document}



\maketitle




% r.5 contents
\tableofcontents


\chapter{Signed Measure}
\section{Introduction}
\begin{defn}[Signed Measure]
	Given a measurable space $(X,\mathcal{M})$, a signed measure is a function $\nu:\mathcal{M}\to[-\infty,\infty]$ with the following properties:
\begin{enumerate}
	\item $\nu(\phi)=0$
	\item $\nu$ can assume either $\infty$ or $-\infty$ but not both 
	\item If $\{E_j\}$ is a sequence of disjoint sets in $\mathcal{M}$, then $\sum_{i=1}^\infty \nu(E_i)=\nu\left(\cup_{i=1}^\infty E_i \right)$
\end{enumerate}
\end{defn}
\marginnote{	
One should note that normal measures are also signed measure, the only difference is the extension of the range of the measure function to cover almost all of $\mathbb R$.
}
\section{Upper and lower continuity}

\begin{theorem}[Uppercontinuity]
	Let $\{E_i\}$ be a countable collection of measurable set with $E_i\subseteq E_{i+1}$. Then:
	\begin{align}
		\lim_{i\to\infty}\nu(E_i)=\nu\left(\bigcup_{i=1}^\infty E_i\right)
	\end{align}
\end{theorem}

\begin{theorem}[Lowercontinuity]
	Let $\{E_i\}$ be a countable collection of measurable set with $E_{i+1}\subseteq E_{i}$. Then:
	\begin{align}
		\lim_{i\to\infty}\nu(E_i)=\nu\left(\bigcap_{i=1}^\infty E_i\right)
	\end{align}
\end{theorem}

\begin{proof}
Same as what we do for unsigned measure
\end{proof}

\section{Positive,Negetive and Null Set}

\begin{defn}[Positive set]
	A set whose every mesurable subset $E$ satisfies $\nu(E)\geq0$ is called a positive set.
\end{defn}
\noindent In a similar fashion we define :

\begin{defn}[Negetive set]
	A set whose every mesurable subset $E$ satisfies $\nu(E)\leq0$ is called a positive set.
\end{defn}

\begin{defn}[Null set]
	A set whose every mesurable subset $E$ satisfies $\nu(E)=0$ is called a positive set.
\end{defn}

\noindent We consider an example. Let $\mu$ be an unsigned measure and let $f$ be a measurable $L^1$ function . Let us define a measure $nu$ as:
\begin{align}
	\nu(E)=\int_Efd\mu
\end{align}
Then $\nu$ is a signed measure. If $E$ is a set such that $f\geq0$ $\mu.a.e$ on $E$ then $E$ is a positive set. Similarly we can find negetive and null sets.

\begin{lemma}
	\begin{enumerate}
		\item Subsets of positive sets are positive
		\item Countable\sidenote{A countable union is needed as in case of uncountable union, there will be a chance that the union will not belong to the sigma algebra; a sigma algebra is closed in countable union and not under arbitary union} union of positive sets are positive
	\end{enumerate}
	Similar results are also valid for null and negetive sets.
\end{lemma}

\noindent The next lemma will be required for the proof of \textbf{Hahn Decomposition Theorem} in the next section.


\begin{lemma}
	Let $\nu$ be a signed measure which doesn't attain $\infty$. A set with a positive measure has a positive subset.
\end{lemma}

\section{Hahn Decomposition}

\begin{theorem}[Hahn Decompositiopn Theorem]
	lf $\nu$ is a signed measure on $(X,\mathcal M)$, there 
	exist a positive set $P$ and a negative set $N$ for $\nu$ such that $P\cup N = X$ and $P\cap N =\phi$ 
	Moreover if  $P', N'$ is another such pair, then $P\Delta P' (= N \Delta N')$ is null in $\nu$.
\end{theorem}
\noindent\textbf{\textit{Proof Outline :}}
\begin{enumerate}
	\item Define $m=\sup_{\text{positive sets}}\nu(P)$
	\item Take a sequence $\{p_i\}$ such that $\lim_{i\to\infty}\nu(p_i)=m$
	\item Show if $P=\bigcup p_i$ then $\nu(P)=m$
	\item Show if $N=P^c$ and if $N$ has a set with positive measure, then by lemma 4, there is contradiction.
	\item If $E\subseteq P\Delta P'$ and $\nu(E)\ne0$. Without loss of generality assume $E'=E\cap P$ is not null. Then $E'\subseteq P'^c=N'$ which contradicts negetivity of $N'$ 
\end{enumerate}
\section{Jordan Decomposition}

\begin{defn}[Mutually singular measures]
Two measures $\nu$ and $\mu$ are said to be mutually singular if there exists a partition of $X$ in $E$ and $F$ such that 	$X=E\sqcup F$ and $E$ is null in $\mu$ and $F$ is null in $\nu$\sidenote{That is to say that the measures $\nu$ and $\mu$ "lives" on different sets.}
\end{defn}\noindent
\textbf{Notation: }If $\nu$ and $\mu$ are mutuallysingular, then we denote it as:
$$\nu\perp\mu$$
\begin{theorem}[Jordan Decomposition Theorem]
Given a (signed) measure $\nu$ there exists \textbf{unique} positive measures $\nu^+,\nu^-$ such that:
\begin{align}
	\nu=\nu^+-\nu^-\quad\nu^+\perp\nu^-
\end{align}	
\end{theorem}\noindent
\textbf{\textit{Proof Outline:}}
\sidenote{As I understand it, the main idea is if there is two decomposition as outlined in step 2 and 3, then we have 4 sets to deal with:
\begin{itemize}
	\item $P\cap F$ and $E\cap N$: which are null as they are intersection of positive and negetive sets
	\item $P\cap E$ where $\mu^+$ and $\nu^+$ agree and $\mu^-,\nu^-=0$
	\item $N\cap F$ where $\mu^-$ and $\nu^-$ agree and $\mu^+,\nu^+=0$
\end{itemize}
Make this nice and you get the proof outlined.
}
\begin{enumerate}
	\item Existance follows by Hahn decomposition.
	\item Start by assuming the decomposition is not unique and theere exists two such decomposition $\nu=\nu^+-\nu^-=\mu^+-\mu^-$.
	\item There exists partition of $X$ in $E,F$ due to $\mu^+,\mu^-$ and in $P,N$ due to $\nu^+,\nu^-$. If $A$ is measurable , show that $$\mu^+(A)=\nu (A\cap E)=\nu(A\cap E\cap P)+\nu(A\cap E\cap N)$$
	\item As $E$ is positive and $N$ is negetive, show that $A\cap E\cap N$ is a null set. Repeat or $\nu^+$ and get similar results 
	\item Show $\nu^+=\mu^+$ and in a similar way $\mu^-=\nu^-$
\end{enumerate}
\section{Total Variation Measure}
\begin{defn}[Total Variation Measure]
	If a measure $\nu$ decomposes in singular $\nu^+$ and $\nu^-$ then we define the total variation measure $|\nu|$ as
	\begin{align}
		|\nu|=\nu^++\nu^-
	\end{align}
\end{defn}\marginnote{This works before as by Hahn-Jordan, the decomposition is unique. The definition is important as by Lemma 7 and Lemma 8, we see that properties of $\nu$ is reflected in $|\nu|$}
\begin{lemma}
	The following statements are equivalent:
	\begin{enumerate}
		\item $E$ is null in $\nu$
		\item $\nu^+(E)=0$ and $\nu^-(E)=0$\sidenote{For unsigned measures, being null and having a measure 0 is same.}
		\item $|\nu|(E)=0$
	\end{enumerate}
\end{lemma}

\begin{lemma}
	The following statemwents are equivalent:
	\begin{enumerate}
		\item $\nu\perp\mu$
		\item $\nu^+\perp\mu$ and $\nu^-\perp\mu$
		\item $|\nu|\perp\mu$
	\end{enumerate}
\end{lemma}
\noindent {Proof for lemma 7 and 8 is  at the  end, they are given as exercise in Folland, ch3}. Other properties which gets reflected are finiteness and $\sigma-$finiteness.
\section{Absolute Continuity}
\begin{defn}[Absolute Continuity]
Let $\mu$ be an unsigned measure. We say $\nu$ is absolutely continious with respect to $\mu$ if for any measurable set $E$, $\mu(E)=0\implies \nu(E)=0$
\end{defn}
\noindent\textbf{Notation:} $\nu$ is absolutely continious with respect to $\mu$ is denoted by:
\begin{align*}
	\nu\ll\mu
\end{align*}
Unlike mutual singulaity, $\nu\ll\mu$ doesn't imply $\mu\ll\nu$. In a sense, being mutually singular and being absolutely continious are exclusive concepts. If $\nu\perp\mu$ and $\nu\ll\mu$ then $\nu=0$

\begin{lemma}
	The following statements are equivalent:
	\begin{enumerate}
		\item $\nu\ll\mu$
		\item $\nu^+\ll\mu$ and $\nu^-\ll\mu$
		\item $|\nu|\ll\mu$ 
	\end{enumerate}
\end{lemma}
\begin{lemma}
	If $\nu$ and $\mu$ are finite measures, $\nu\ll\mu$ if and only if for every $\epsilon>0$ there exists $\delta>0$ usch that $|\nu(E)|<\epsilon$ whenever $\mu(E)<\delta$ \sidenote{This lemma gives some motivation for the nomenclature of  absolute continuity }
\end{lemma}\noindent
\textit{\textbf{Proof Outline: }}
\begin{enumerate}
	\item By Lemma 9, we need to show this is true for $|\nu|$ and we will be done. This is why, without loss of generality, we can assume $\nu$ is unsigned.
	\item \textbf{\textit{Don't understand why this is trivial}}
	\item Make a decresing sequence of mesurable sets
	\item Show if there exists $\epsilon$ with no such $\delta$ then $\mu$ of intersection goes to 0 but $\nu$ of intersection stays above $\epsilon$. This contradicts absolute continuity.
\end{enumerate}
\section{Radon-Nikodym theorem}
\begin{theorem}[Radon-Nikodym theorem]
	The theorem has two parts:
	\begin{enumerate}
		\item For a measure space, with $\sigma-$finite measures $\nu$(unsigned) and $\mu$(unsigned), there is a unique decomposition of $\nu$ in $\nu_1$ and $\nu_2$ such that $\nu_1\ll\mu$ and $\nu_2\perp\mu$
		\item There exists a function $f$ which is integrable in the etended sense such that $\nu_1(E)=\int_Efd\mu$. Moreover, if there are two such functions $f_1,f_2$ then $f_1=f_2$ $\mu.a.e$. 
	\end{enumerate}
\end{theorem}
\textbf{\textit{Proof Outline:}}\\
\begin{enumerate}
	\item \textbf{Step 1: $\nu$, $\mu$ are finite}\\
		\begin{enumerate}
			\item Note that $\nu(E)=\int_Efd\mu+\nu_2(E)\Rightarrow \nu(E)\geq\int_Efd\mu$
			\item Make a family of function $\mathcal{F}$ which satisfy this. 
			\item Let $\alpha$ be suprema of the integral of $f$ in family. Find $f_n$ whose integral approach $\alpha$. Set $g_n(x)=\max\{f_1(x),f_2(x)\hdots f_n(x)\}$. Show $g_n$ is increasing and is in $\mathcal{ F}$. Find limit of $g_n$ as $g$. Use MCT to show that $\alpha$ is attained by $g$.\sidenote{The reason to take a in first place is because we don't know if $f_n$ converges. The reason we take such a $g_n$ is sothat we can intechange the limit and integral by applying MCT.}
			\item  Set $\nu_2=\nu-\nu_1$. Show $\nu_2\perp\mu$. 
		\end{enumerate}
	\item  \textbf{Step 2: Assume $\sigma,\mu$ are $\sigma-$finite.}
	\begin{enumerate}
		\item Divide $X$ in disjoint countable $B_i$ each with finite measure.
		\item Restrict $\mu$ and $\nu$ in $B_i$ to get $\mu_i,\nu_i$. Repeat step 1 to get $f_i,\nu^1_i,\nu^2_i$ in $B_i$. Set $f=\sum f_i,\nu^1=\sum\nu^1_i,\nu^2=\sum\nu^2_i$
	\end{enumerate} 
	\item \textbf{Step 3: Uniqueness of decomposition}:
	If $\nu_1,\nu_2$ and $\hat\nu_1,\hat\nu_2$ are two decomposition then $\nu_1-\hat\nu_1=\hat\nu_2-\nu_2$. Now $(\nu_1-\hat\nu_1)\ll\mu$ and $\nu_2-\hat\nu_2\perp\mu$. So $\nu_1-\hat\nu_1=0$
	\item \textbf{Step 4: Uniqueness of $f$ :}
	If $f,g$ are two such functions then $\int_E(f-g)=0$ or $f=g$ $\mu.a.e$.  	
\end{enumerate}


\section{Solutions to Real Analysis By Folland, Section 3.1,3.2}
\subsection{Solutions to problems in section 3.1}
\begin{tcolorbox}[colback=c4,colframe=c3,title=Problem 3.1]
	Prove Proposition 3.1.\\\noindent
	{Propotion 3.1} is Theorem 1\&2 mentioned here.
\end{tcolorbox}\noindent
\textbf{Solution 1 outline:} We do it as instructed in the book, by copying Theorem 1.8 from the book.\\
\textbf{Solution 2 outline:} Decompose $\mu$ by Hahn Decomposition and apply upper/lower continuity on each of them  individually.\sidenote{Proof of Hahn decomposition doesn't assume upper/lower continuity}\\
\textbf{Proof 1:}\\\noindent
By the second condition of definition 1, we can assume $\mu>-\infty$.\\ 
\noindent \textbf{Proving Uppercontinuity:} If some $E_i=\infty$ we are done. Else , set $E_0=\phi$. Define $F_i=E_i\setminus E_{i-1}$. Not that any two $F_i,F_j$ is disjoint. Then $\cup_{i=1}^\infty E_i= \cup_{i=1}^\infty F_i$. It follows that:
\begin{align*}
	\mu\left(\bigcup_{i=1}^\infty E_i\right)=&\mu\left(\bigcup_{i=1}^\infty F_i\right)
	=\sum_{i=1}^\infty \mu\left(F_i\right)=\lim_{i\to\infty}\mu(E_i)
\end{align*}
The last step follows by countable additivity.\\\noindent
\noindent \textbf{Proving Lowercontinuity:}
Set $F_j=E_1\setminus E_j$. Then $F_i\subseteq F_{i+1}$ and $\mu(E_1)=\mu(F_j)+\mu(E_j)$. Also, $\cup_{i=1}^\infty F_j=E_1\setminus(\cap_{i=1}^\infty E_j)$. Apply uppercontinuity to get: 
\begin{align*}
	\mu(E_1)=\mu\left(\bigcap_{i=1}^\infty E_j\right)+\lim_{j\to\infty}\mu(F_j)=\mu\left(\bigcap_{i=1}^\infty E_j\right)+\mu(E_1)-\lim_{j\to\infty}\mu(E_j)
\end{align*}
\begin{align*}
	\Rightarrow \lim_{j\to\infty}\mu(E_j)=\mu\left(\bigcap_{i=1}^\infty E_j\right)
\end{align*}

\begin{tcolorbox}[colback=c4,colframe=c3,title=Problem 3.2]
	If $\nu$ is a signed measure, $E$ is $\nu$-null iff  $|\nu|(E)=0$. Also, if $\nu$ and $\mu$ are signed 
measures, $\nu\perp\mu$ iff  $\nu^+\perp\mu$ and $\nu^-\perp\mu$
\end{tcolorbox}\noindent
\textbf{Solution outline:} This the the problem corresponding to lemma 7 and 8. In both case we shall show
$1\Rightarrow2\Rightarrow3\Rightarrow1$.\\
\textbf{Solution part-1(Proof of lemma 7): }\\
The Steps are based on lemma 7.
\begin{itemize}
	\item \textbf{Step 1: $1\Rightarrow 2$}\\
		Let $P,N$ be the decomposition of $N$ in positive and negetive sets using HJD[Hahn Jordan Decomposition].  Let $E$ be a null set. Then:
		\begin{align*}
			\nu^+(E)=&\nu^+(E\cap P)-\nu^-(E\cap N)\\
			=&\nu^+(E\cap P)\\
			=&\nu(E\cap P)-\nu^-(E\cap P)\\
			=&\nu(E\cap P)=0
		\end{align*}
		Similar reult is obtained for $\nu^-$. For unsigned measures, a measure zero set is null set, so we are done.
	\item \textbf{Step 2: $2\Rightarrow 3$}\\
	This is the easy step.
	\begin{align*}
		|\nu|(E)=\nu^+(E)+\nu^-(E)=0
	\end{align*}
	\item \textbf{Step 3: $3\Rightarrow 1$}\\
	Note that for any measurable subset $A$ of $E$ we have $|\nu|(A)=0$. We also have:
	$$|\nu(A)|=|\nu^+(A)-\nu^-(A)|\leq \nu^+(A)+\nu^-(A)=|\nu|(A)=0$$
	$$\Rightarrow \nu(A)=0$$
	Therefore, $E$ is null in $\nu$
\end{itemize}
\textbf{Solution part-2(Proof of lemma 8): }\\
The Steps are based on lemma 8.
\begin{itemize}
	\item \textbf{Step 1: $1\Rightarrow 2$}\\
	Let $P,N$ be the decomposition of $N$ in positive and negetive sets using HJD[Hahn Jordan Decomposition]. Let $A,B$ be the disjoint decomposition of $X$ for $\nu$ and $\mu$. Then it is easy to check every element lies in one of the four sets: $A\cap P,A\cap N,B\cap P,B\cap N$. Now note,
	\begin{itemize}
		\item Decomposition of $X$ for $\nu^+$ and $\mu$ is achived by $A\cap P$ and $(A\cap N)\cup (B\cap P)\cup(B\cap N)$
		\item Decomposition of $X$ for $\nu^-$ and $\mu$ is achived by $A\cap N$ and $(A\cap P)\cup (B\cap P)\cup(B\cap N)$
	\end{itemize} 
	\item  \textbf{Step 2: $2\Rightarrow 3$}\\
	Let decomposition of $X$ for $\nu^+$ and $\mu$ be $E_1,F_1$ and for $\nu^-$ and $\mu$ be $E_2,F_2$. We calim the decomposition of $X$ for $|\nu|$ and $\mu$ is given by $E_1\cup E_2$ and $F_1\cap F2$.\sidenote{This is low-key motivated by the decomposition in step 1.}
	Note that $|\nu|$ is null in $F_1\cap F_2$ both $\nu^+$ and $\nu^-$ is null in $F_1,F_2$. $\mu$ is null in both $E_1$ and $E_2$. By lemma 3, part 2, $\mu$ is null in $E_1\cup E_2$.
	\item   \textbf{Step 3: $3\Rightarrow 1$}\\
	Let $A,B$ be the disjoint decomposition of $X$ for $|\nu|$ and $\mu$. We claim this is the appropiate decomposition for $\nu$ and $\mu$ as well. It is already known $\mu$ is null in $A$. As $|\nu$ is null in $B$, $\nu$ is null in $B$ follows from lemma 7($3\Rightarrow1$). 
\end{itemize}



\begin{tcolorbox}[colback=c4,colframe=c3,title=Problem 3.3]
	Let $\nu$ be a signed measure on $(X,\mathcal M)$. Prove
	\begin{enumerate}
		\item $\mathcal L^1(\nu)=\mathcal L^1(|\nu|)$
		\item If $f\in\mathcal{L}^1(\nu)$, $|\int fd\nu|\leq \int|f|d|\nu|$
		\item If $e\in\mathcal M$, $|\nu|(E)|=\sup \{|\int_Efd\nu|:|f|\leq 1\}$
	\end{enumerate}
\end{tcolorbox}
\textbf{Solution outline:} Our main goal will be to study $f$ on the decomposition of $X$ made by $HJD$ due to $\nu$\\
\textbf{Solution part-1:}\\
Let $X$ be decomposed into positive set $P$ and negetive set $N$. We assume $\nu>-\infty$. Let $f\in\mathcal{L}^1(\nu)$. Let $\chi_E$ denote the characteristic function on $E$. Then we have:
\begin{align}
	\int|f|d|\nu|=\int|f|(\chi_P+\chi_N)d(\nu^++\nu^-)=\int|f|d\nu^++\int|f|d\nu^-<\infty
\end{align}
Therefore, $f\in\mathcal L^1(|\nu|)$.\\

Now assume $f\in\mathcal L^1(|\nu|)$. Then as before,
\begin{align}
	\infty>\int|f|d|\nu|=\int|f|(\chi_P+\chi_N)d(\nu^++\nu^-)=\int|f|d\nu^++\int|f|d\nu^-
\end{align}
But as $\nu+,\nu^-$ are both unsigned we can conclude that $\int|f|d\nu^+,\int|f|d\nu^-<\infty$. Therefore, $f\in\mathcal L^1(\nu)$\\
\textbf{Solution part-2:}\\
\begin{align*}
	\left|\int fd\nu\right|=&\left|\int f(\chi_P+\chi_N)d(\nu^+-\nu^-)\right|\\
	=&\left|\int f d\nu^+-\int f d\nu^- \right|\\
	\leq&\left|\int f d\nu^+\right|+\left|\int f d\nu^- \right|\\
	\leq&\int |f| d\nu^++\int |f| d\nu^-=\int|f|d|\nu|
\end{align*}
\textbf{Solution part-2:}\\
Firt we show that $|\nu|(E)$ is an upper bound and then we show that it is attained. For any measurable $f$ with $|f|\leq1$ we have:
$$\left|\int_Efd\nu\right|\leq \int_E|f|d|\nu|\leq \int_Ed|\nu|=|\nu|(E)$$
Therefore, $|\nu|(E)$ is an upper bound. Now set $f=\chi_P-\chi_N$. For any $x\in X$,either  $x\in P$ or $x\in N$. Therefore $f(x)\in\{1,-1\}$.  
\begin{align*}
	\left|\int f_Ed\nu\right|=&\left|\int_E (\chi_P-\chi_N)(\chi_P+\chi_N)d(\nu^+-\nu^-)\right|\\
	=&\left|\int_E(\chi_P^2-\chi_N^2)d(\nu^+-\nu^-)\right|\\
	=&\left|\int_E\chi_P^2d\nu^++\int_E\chi_N^2d\nu^-\right|\\
	=&\left|\int_E\chi_Pd\nu^++\int_E\chi_Nd\nu^-\right|\\
	=&|\nu^+(E\cap P)+\nu^+(E\cap N)|\\
	=&|\nu^+(E\cap P)+\nu^+(E\cap N)+\nu^+(E\cap N)+\nu^+(E\cap P)|\\
	=&\nu^+(E)+\nu^-(E)=|\nu|(E)
\end{align*}

\backmatter

\bibliography{sample-handout}
\bibliographystyle{plainnat}


\printindex

\end{document}

