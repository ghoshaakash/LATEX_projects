    \documentclass[a4paper, 11pt]{article}
\usepackage{comment} % enables the use of multi-line comments (\ifx \fi) 
\usepackage{lipsum} %This package just generates Lorem Ipsum filler text. 
\usepackage{fullpage} % changes the margin
\usepackage[a4paper, total={7in, 10in}]{geometry}
\usepackage{amsmath}
\usepackage{amssymb,amsthm}
\usepackage{graphicx}
\usepackage{tikz}
\usetikzlibrary{arrows}
\usepackage{verbatim}
\usepackage[numbered]{mcode}
\usepackage{float}
\usepackage{tikz}
    \usetikzlibrary{shapes,arrows}
    \usetikzlibrary{arrows,calc,positioning}

    \tikzset{
        block/.style = {draw, rectangle,
            minimum height=1cm,
            minimum width=1.5cm},
        input/.style = {coordinate,node distance=1cm},
        output/.style = {coordinate,node distance=4cm},
        arrow/.style={draw, -latex,node distance=2cm},
        pinstyle/.style = {pin edge={latex-, black,node distance=2cm}},
        sum/.style = {draw, circle, node distance=1cm},
    }
\usepackage{xcolor}
\usepackage{mdframed}
\usepackage[shortlabels]{enumitem}
\usepackage{indentfirst}
\usepackage{hyperref}
\usepackage{mhchem}
\renewcommand{\thesubsection}{\thesection.\alph{subsection}}
\usepackage{tcolorbox}
\newenvironment{problem}[2][Problem]
    { \begin{mdframed}[backgroundcolor=gray!20] \textbf{#1 #2} \\}
    {  \end{mdframed}}

% Define solution environment
\newenvironment{solution}
    {\textit{Solution:}}
    {}



    \newtheorem{defn}{Definition}
    \newtheorem{theorem}{Theorem}
    \newtheorem{corollary}{Corollary}[theorem]
    \newtheorem{lemma}[theorem]{Lemma}
    \newtheorem{remark}{Remark}
    
    \definecolor{c1}{HTML}{1C3879} 
    \definecolor{c2}{HTML}{F9F5EB} 

\renewcommand{\qed}{\quad\qedsymbol}
%%%%%%%%%%%%%%%%%%%%%%%%%%%%%%%%%%%%%%%%%%%%%%%%%%%%%%%%%%%%%%%%%%%%%%%%%%%%%%%%%%%%%%%%%%%%%%%%%%%%%%%%%%%%%%%%%%%%%%%%%%%%%%%%%%%%%%%%
\begin{document}
%Header-Make sure you update this information!!!!
\noindent








%%%%%%%%%%%%%%%%%%%%%%%%%%%%%%%%%%%%%%%%%%%%%%%%%%%%%%%%%%%%%%%%%%%%%%%%%%%%%%%%%%%%%%%%%%%%%%%%%%%%%%%%%%%%%%%%%%%%%%%%%%%%%%%%%%%%%%%%
\large\textbf{Aakash Ghosh} \hfill \textbf{}   \\
Email: ag19ms129@iiserkol.ac.in  \hfill ID: 19MS129 \\
\normalsize Course: Algebra III \hfill Term: Autumn 2022\\
\noindent\rule{7in}{2.8pt}
%%%%%%%%%%%%%%%%%%%%%%%%%%%%%%%%%%%%%%%%%%%%%%%%%%%%%%%%%%%%%%%%%%%%%%%%%%%%%%%%%%%%%%%%%%%%%%%%%%%%%%%%%%%%%%%%%%%%%%%%%%%%%%%%%%%%%%%%
% Problem 1
%%%%%%%%%%%%%%%%%%%%%%%%%%%%%%%%%%%%%%%%%%%%%%%%%%%%%%%%%%%%%%%%%%%%%%%%%%%%%%%%%%%%%%%%%%%%%%%%%%%%%%%%%%%%%%%%%%%%%%%%%%%%%%%%%%%%%%%%
\section{Problem Sheet 1}
\begin{tcolorbox}[colback=c2,colframe=c1,title=Problem 1.5]
    Let $F \hookrightarrow K$ be a field extension and $a, b \in K$ be algebraic over $F$ with
    minimum polynomials of degree $p, q$, where $p, q$ are distinct prime numbers.
    Show that $[F(a, b) : F] = pq$.
\end{tcolorbox}
\begin{solution}
    Note that as $[F(a):F]\mid[F(a,b):F]$ and $[F(b):F]\mid[F(a,b):F]$, $[F(a,b):F]$ is of the form $kpq$. Let $\{\alpha_i\}_{1\leq i\leq p}$ be a basis of $F(a)\mid_F$ and $\{\beta_j\}_{1\leq j\leq q}$ be a basis of $F(b)\mid_F$. Consider the field $F'$ spanned by $\{\alpha_i\beta_j\}_{1\leq i\leq p,1\leq j\leq q}$. $a,b\in F'$ and therefore $F(a,b)\subseteq F'$. By our construction $[F':F]\leq pq$. Therefore, $[F(a,b):F]\leq pq$. It follows that $k=1$ which completes the proof.
\end{solution}

\begin{tcolorbox}[colback=c2,colframe=c1,title=Problem 1.1]
    Determine
\begin{enumerate}
    \item $[\mathbb{Q}(\sqrt2,\sqrt5),\mathbb{Q}]$
    \item $[\mathbb{Q}(\sqrt2,\sqrt5,\sqrt7),\mathbb{Q}]$
    \item $[\mathbb{Q}(\sqrt[3]2,\sqrt3),\mathbb{Q}]$
\end{enumerate}
\end{tcolorbox}\noindent
\begin{solution}
    We shall use the result of problem 1.5 proved above. 
\begin{lemma}
    If $p$ is prime, then $x^n-p$ is irreducible over $\mathbb Q$ for $n>1$.
\end{lemma}
\begin{proof}
    By applying Einstein's criteria over $p$ on $x^n-p$ we get $x^n-p$ is irreducible over $\mathbb Z$. By Gauss lemma, we conclude that $x^n-p$ is irreducible over $\mathbb Q$ too.
\end{proof}
\begin{enumerate}
    \item Note that $\sqrt 2$ satisfies $x^2-2=0$. By lemma 1, this polynomial is irreducible and therefore $[\mathbb Q(\sqrt 2):\mathbb Q]=2$. Note that $x^2-5=0$ has $\sqrt 5$ as root. Therefore, the minimal polynomial  of $\sqrt 5$ over $\mathbb Q(\sqrt 2)$, $p(x)$, divides $(x^2-5)$. We show $p$ is not linear, i.e. $\sqrt5\notin\mathbb Q(\sqrt2)$. Assume to the contrary and let $\sqrt 5=a+b\sqrt 2$ where $a,b\in\mathbb Q$. Then we note:
    \begin{align*}
        &\sqrt 5=a+b\sqrt 2\\
        \Rightarrow&5=a^2+2b^2+2ab\sqrt2\\
        \Rightarrow&\sqrt2=\frac{5-a^2-2b^2}{2ab}\in\mathbb{Q}
    \end{align*}
    Which is a contradiction. Therefore, $p(x)$ has degree 2. It follows that
    $$[\mathbb Q(\sqrt 2,\sqrt5):\mathbb Q]=[\mathbb Q(\sqrt 2,\sqrt5):\mathbb Q(\sqrt 2)]\times [\mathbb Q(\sqrt 2):\mathbb Q]=4$$
    \item Like before, note that $x^2-7=0$ has $\sqrt 7$ as root. Therefore, the minimal polynomial  of $\sqrt 7$ over $\mathbb Q(\sqrt 2,\sqrt 5)$, $p(x)$, divides $(x^2-7)$. Note that $\{1,\sqrt 2\}$ is a basis of $\mathbb Q(\sqrt 2)$ over $\mathbb Q$ and $\{1,\sqrt 5\}$ is a basis of $\mathbb Q(\sqrt 5)$ over $\mathbb Q$. Therefore, as we pointed in Problem 1.5, $\{1,\sqrt2,\sqrt5,\sqrt{10}\}$ is a basis of $\mathbb Q(\sqrt 2,\sqrt5)$ over $\mathbb Q$. We shall show $\sqrt 7\notin \mathbb Q(\sqrt 2,\sqrt 5)$. Assume to the contrary and let $\sqrt{7}=a+b\sqrt2+c\sqrt5+d\sqrt{10}$. Then:
    \begin{align*}
        &\sqrt 7=a+b\sqrt2+c\sqrt5+d\sqrt{10}\\
        \Rightarrow&7=(a^2+2b^2+5c^2+10d^2)+2ab\sqrt2+\left(2ac+2(bc+ad)\sqrt2\right)\sqrt 5\\
        \Rightarrow&\sqrt5=\frac{7-\left((a^2+2b^2+5c^2+10d^2)+2ab\sqrt2\right)}{2ac+\left(2bc+ad\right)\sqrt2}\in\mathbb{Q}(\sqrt 2)
    \end{align*}
    Which is a contradiction. Therefore, degree of $p$ is two and 
    $$[\mathbb Q(\sqrt 2,\sqrt5,\sqrt 7):\mathbb Q]=[\mathbb Q(\sqrt 2,\sqrt5,\sqrt7):\mathbb Q(\sqrt 2,\sqrt 5)]\times [\mathbb Q(\sqrt 2,\sqrt 5):\mathbb Q]=8$$
    \item  By Lemma 1, $x^2-3$ and $x^3-2$ are irreducible. As $\sqrt3,\sqrt[3]2$ are roots of those polynomials, $[\mathbb Q(\sqrt 3):\mathbb Q]=2,[\mathbb Q(\sqrt[3] 2):\mathbb Q]=3$. By Problem 1.5, it follows that $[\mathbb{Q}(\sqrt[3]2,\sqrt3),\mathbb{Q}]=2\times3=6$
\end{enumerate}
\end{solution} 







\begin{tcolorbox}[colback=c2,colframe=c1,title=Problem 1.2]
    Prove that finite fields can't be algebraically closed.
\end{tcolorbox}
\begin{solution}
    Let $\mathbb F$ be a finite field. Consider the polynomial $p(x)=\prod_{\alpha\in\mathbb F}(x-\alpha)+1$. This polynomial has no roots in $\mathbb F$. Consider the splitting field $\mathbb F'$ of $p$ over $\mathbb F$. Therefore, there exists a proper algebraic extension of $\mathbb F$ and $\mathbb F$ is not closed.
\end{solution}




\begin{tcolorbox}[colback=c2,colframe=c1,title=Problem 1.3]
    Prove that any extension of prime order (i.e., the degree is prime) is simple.
\end{tcolorbox}
\begin{solution}
    Let $K$ be a prime order extension of $F$. Let $\alpha\in K$ such that $\alpha\notin F$. Since ${[F(\alpha):F]\mid[K:F(\alpha)]=p}$, it follows that $[F(\alpha):F]$ is either $1$ or $p$. But as $\alpha\notin F$ it follows that $[F(\alpha):F]=p$. Therefore, 
    ${[K:F(\alpha)]=\frac{[K:F]}{[F(\alpha):F]}}=1$. Therefore, $K=F(\alpha)$ and $K$ is simple. 
\end{solution}




\begin{tcolorbox}[colback=c2,colframe=c1,title=Problem 1.4]
    Is $\mathbb{Q}(\sqrt2,\sqrt5)$ simple?
\end{tcolorbox}
\begin{solution}
    Yes. We show $\mathbb{Q}(\sqrt2,\sqrt5)=\mathbb{Q}(\sqrt 2+\sqrt 5)$. It is easy to see that $\mathbb{Q}(\sqrt 2+\sqrt 5)\subseteq \mathbb{Q}(\sqrt2,\sqrt5)$ as $\sqrt2+\sqrt 5\in \mathbb{Q}(\sqrt2,\sqrt5)$. Let $p(x)$ be the minimal polynomial of $\sqrt2+\sqrt 5$ over $\mathbb Q$. Now as $\sqrt2+\sqrt 5$ is irrational, $p$ is not linear. Assume $p$ is quadratic and there exists rational $b,c$ such that $x^2+bx+c$ has $\sqrt2+\sqrt 5$ as roots. Then on putting $x=\sqrt2+\sqrt5$ we get:
    \begin{align*}
        &(\sqrt 2+\sqrt5)^2+b(\sqrt2+\sqrt5)+c=0\\
        \Rightarrow& 7+c+b\sqrt2+(b+2\sqrt2)\sqrt5=0\\
        \Rightarrow&\sqrt5=-\frac{7+c+b\sqrt2}{b+2\sqrt2}\in\mathbb{Q}\sqrt2
    \end{align*}
    Which is not possible as shown in problem 1.1(part 1). Therefore, $p$ is not quadratic. Now we have $deg(p)=[\mathbb Q(\sqrt2+\sqrt5):\mathbb Q]>2$. By problem 1.1 we have $[\mathbb Q(\sqrt2,\sqrt5):\mathbb Q]=4$. As $[\mathbb Q(\sqrt2+\sqrt5):\mathbb Q]\mid[\mathbb Q(\sqrt2,\sqrt5):\mathbb Q]$, the only possible value of $[\mathbb Q(\sqrt2+\sqrt5):\mathbb Q]$ is  4. It follows that $[\mathbb Q(\sqrt2+\sqrt5):\mathbb Q(\sqrt2,\sqrt5)]=[\mathbb Q(\sqrt2,\sqrt5):\mathbb Q]/[\mathbb Q(\sqrt2+\sqrt5):\mathbb Q]=1$. Therefore, $\mathbb{Q}(\sqrt2,\sqrt5)=\mathbb{Q}(\sqrt 2+\sqrt 5)$ and $\mathbb{Q}(\sqrt 2,\sqrt 5)$ is simple.
\end{solution}

\begin{tcolorbox}[colback=c2,colframe=c1,title=Problem 1.6]
    Show that an extension $F\hookrightarrow K $is algebraic if and only if every ring $R$ with
$F \subseteq R \subseteq K$ is a field.
\end{tcolorbox}
\begin{solution}
    Let $K$ be algebraic. We show that $R$ is a field. To show this we only need to show that if $\alpha\in R$, $\alpha^{-1}\in R$. As $K$ is algebraic, there exists some polynomial $p(x)$ such that $p(\alpha)=0$. Let $p=\sum_{i=0}^na_ix^i$ where $a_i\in F$. We assume $p$ is irreducible and therefore $a_0\ne0$. Then
    \begin{align*}
        &\sum_{i=0}^na_i\alpha^i=0\\
        \Rightarrow&\sum_{i=1}^na_i\alpha^i=-a_0\\
        \Rightarrow&\alpha\left(\sum_{i=1}^na_i\alpha^{i-1}\right)=-a_0\\
        \Rightarrow&\alpha\left[\frac{1}{-a_0}\left(\sum_{i=1}^na_i\alpha^{i-1}\right)\right]=1\\
    \end{align*}
    Therefore, $\alpha$ has an inverse and $R$ is a field.\\\\
    Now we assume all such $K$ are fields. We show $F$ is algebraic. We prove by contradiction. Assume there exists a transcendental  element $\alpha$. Then $F[\alpha]\equiv F[X]$. Consider the ring $F[X]$. We show that this is not a field. Let $p(x)$ be a non-zero polynomial with an inverse $q(x)$ over $K$. Then:
    $p(\alpha)q(\alpha)=1\Rightarrow p(\alpha)q(\alpha)-1=0$. Therefore, $pq-1$ has $\alpha$ as a root which contradict the fact that $\alpha$ is transcendental. Therefore, no such $\alpha$ exist and  $F$ is algebraic.
\end{solution}



\begin{tcolorbox}[colback=c2,colframe=c1,title=Problem 1.7]
    Let $F\hookrightarrow K$  be a field extension and $a \in K$ with its minimal polynomial of
degree $m$. Show that $m\mid[K : F]$.
\end{tcolorbox}
\begin{solution}
    We note that $m=[F(a):F]$ and by tower lemma: $[K:F]=[K:F(a)][F(a):F]$. The solution follows. 
\end{solution}




\begin{tcolorbox}[colback=c2,colframe=c1,title=Problem 1.8]
    Does there exist a polynomial $f(X) \in \mathbb Z[X]$ of degree at least 2 such that
$f(X)$ is irreducible over $\mathbb Z_p[X]$ for each prime $p$?
\end{tcolorbox}
\begin{solution}
    No. Without loss of generality assume leading term of $f$ is positive and GCD of all coefficients of $f=1$. 
    Then $\lim_{x\to\infty}f(x)=\infty$. Therefore, there exist some natural $n$ such that $f(n)=\alpha>1$. Let $p$(which can be $\alpha$ itself) be a prime factor of $\alpha$. Let denote $[x]$ to be the equivalence class of $x$ such that $x\equiv[x](\mod p)$. If $f(x)=\sum_{i=1}^rc_ix^i$, then note that:
    $$[f]\left([n]\right)=\sum_{i=1}^r\left[c_i\right][n]^i=\left[\sum_{i=1}^rc_in^i\right]=[\alpha]=0$$
    Therefore, $f$ has a root in $\mathbb Z/p\mathbb Z$ and is reducible.
\end{solution}





\begin{tcolorbox}[colback=c2,colframe=c1,title=Problem 1.9]
    Let $\overline{ \mathbb Q} := \{x \in \mathbb C \mid x$ is algebraic over $\mathbb Q\}$. Is $[\overline{\mathbb Q}\mid\mathbb Q]$ finite?
\end{tcolorbox}
\begin{solution}
    Let $p$ be a prime. Consider the equation $f_p(x)=x^p+p$. Let $\alpha_p$ be the root. It follows minimal polynomial of $\alpha_p$ divides $f_p$. But $f_p$ is irreducible by Einstein's criteria and Gauss's lemma. Therefore, minimal polynomial of $\alpha_p$ is $f_p$ and $[\mathbb Q(\alpha_p):\mathbb{Q}]=p$. As $\alpha_p$ is algebraic, $\alpha_p\in\overline{\mathbb Q}$. Therefore, $\mathbb Q\hookrightarrow\mathbb Q(\alpha_p)\hookrightarrow \overline{\mathbb Q}$ and $p\mid[\overline{\mathbb Q}:\mathbb Q]$. As this is true for all primes $p$ and as there are infinite primes, $[\overline{\mathbb Q}:\mathbb Q]$ is not finite.
\end{solution}

\begin{tcolorbox}[colback=c2,colframe=c1,title=Problem 1.10]
    Prove that there exists a field $F$ such that\\
(a) $F$ is infinite,\\
(b) $F$ is algebraic over a finite field, and\\
(c) $F$ is not algebraically closed
\end{tcolorbox}
\begin{solution}
    
\end{solution}





\section{Problem sheet 2}


\begin{tcolorbox}[colback=c2,colframe=c1,title=Problem 2.1]
    Find the degree of the splitting fields.
    \begin{enumerate}
        \item $f(X)=X^4-2\in\mathbb Q(X)$ over $\mathbb Q$
        \item $f(X)=X^4+1\in\mathbb Q(X)$ over $\mathbb Q$
    \end{enumerate}
\end{tcolorbox}
\begin{solution}
    \begin{enumerate}
        \item The roots of $f$ are $\pm\sqrt[4]{2},\pm i\sqrt[4]{2}$. The splitting field contains $\sqrt[4]{2}$ and $i$. The smallest such field is $\mathbb{Q}(\sqrt[4]{2},i)$. By applying Einstein's criteria on $f$ we conclude $f$ is irreducible. As $\sqrt[4]{2}$ is a root of $f$, we conclude $f$ is minimal polynomial of $\sqrt[4]{2}$ and $[\mathbb{Q}(\sqrt[4]{2}):\mathbb{Q}]=4$. Now as $i$ satisfies $p(x)=x^2+1$ and $i\notin \mathbb{Q}(\sqrt[4]{2})$, we conclude $p$ is minimal polynomial of $i$ over $\mathbb{Q}(\sqrt[4]{2})$. Therefore:
        $$\text{order of splitting field}=[\mathbb{Q}(\sqrt[4]{2},i),\mathbb{Q}]=[\mathbb{Q}(\sqrt[4]{2},i),\mathbb{Q}(\sqrt[4]{2})][\mathbb{Q}(\sqrt[4]{2}):\mathbb{Q}]=8$$
        \item Let the splitting field be $F$. The roots of $f$ are $\pm\frac{1}{\sqrt{2}}\pm i\frac{1}{\sqrt{2}}$. By adding the roots pairwise we get $\sqrt{2},i\in F$. The smallest such field is $\mathbb{Q}[\sqrt2,i]$. It is shown before that $[\mathbb{Q}(\sqrt{2}):\mathbb{Q}]=2$. As $i$ satisfies $p(x)=x^2+1$ and $i\notin \mathbb{Q}(\sqrt{2})$, we conclude $p$ is minimal polynomial of $i$ over $\mathbb{Q}(\sqrt{2})$. So we have $[\mathbb{Q}(\sqrt{2},i):\mathbb{Q}(\sqrt{2})]=2$ Therefore:
        $$\text{order of splitting field}=[\mathbb{Q}(\sqrt{2},i),\mathbb{Q}]=[\mathbb{Q}(\sqrt{2},i),\mathbb{Q}(\sqrt{2})][\mathbb{Q}(\sqrt{2}):\mathbb{Q}]=4$$
    \end{enumerate}
\end{solution}




\begin{tcolorbox}[colback=c2,colframe=c1,title=Problem 2.2]
    Let $K \mid F$ be an extension of degree 2. Show that $K \mid F$ is normal.
\end{tcolorbox}
\begin{solution}
    Let for $\alpha\in K$, $p$ be the minimal polynomial of $\alpha$ over $F$. Then $p$ has degree 1 or 2. If $p$ has degree 1 then $p$ has a single root which is $\alpha$, and we are done. Else let $p(x)=x^2-b x+c=(x-\alpha)(x-\beta)$, where $\beta$ is the other root of $p$. We need to show $\beta\in K$. Direct expansion shows that  $b=\alpha+\beta\Rightarrow\beta=b-\alpha$. As $K$ is a field, $\beta\in K$ which completes the proof. 
\end{solution}

\begin{tcolorbox}[colback=c2,colframe=c1,title=Problem 2.3]
    Prove that $\mathbb Q(\sqrt 2+\sqrt{3})|\mathbb{Q}$ is a normal extension.
\end{tcolorbox}
\begin{solution}
    \begin{lemma}
        $\sqrt3\notin \mathbb Q(\sqrt{2})$
    \end{lemma}
    \begin{proof}
        Assume to the contrary and let $\sqrt 3=a+b\sqrt 2$ where $a,b\in\mathbb Q$. Then we note:
    \begin{align*}
        &\sqrt 3=a+b\sqrt 2\\
        \Rightarrow&3=a^2+2b^2+2ab\sqrt2\\
        \Rightarrow&\sqrt2=\frac{3-a^2-2b^2}{2ab}\in\mathbb{Q}
    \end{align*}
    Which is a contradiction.
    \end{proof}
    \begin{lemma}
        $[\mathbb Q(\sqrt 2,\sqrt{3}):\mathbb{Q}=4]$
    \end{lemma}
    \begin{proof}
        As shown in problem 1.1,  $[\mathbb Q(\sqrt 2):\mathbb Q]=2$. Note that $x^2-3=0$ has $\sqrt 3$ as root. Therefore, the minimal polynomial  of $\sqrt 3$ over $\mathbb Q(\sqrt 2)$, $p(x)$, divides $(x^2-3)$. We have shown $p$ is not linear, i.e. $\sqrt3\notin\mathbb Q(\sqrt2)$. Therefore, $p(x)$ has degree 2. It follows that
    $$[\mathbb Q(\sqrt 2,\sqrt3):\mathbb Q]=[\mathbb Q(\sqrt 2,\sqrt3):\mathbb Q(\sqrt 2)]\times [\mathbb Q(\sqrt 2):\mathbb Q]=4$$
    \end{proof}
    \begin{lemma}
        $\mathbb{Q}(\sqrt2,\sqrt3)=\mathbb{Q}(\sqrt 2+\sqrt 3)$
    \end{lemma}
    \begin{proof}
        It is easy to see that $\mathbb{Q}(\sqrt 2+\sqrt 3)\subseteq \mathbb{Q}(\sqrt2,\sqrt3)$ as $\sqrt2+\sqrt 3\in \mathbb{Q}(\sqrt2,\sqrt3)$. Let $p(x)$ be the minimal polynomial of $\sqrt2+\sqrt 3$ over $\mathbb Q$. Now as $\sqrt2+\sqrt 3$ is irrational, $p$ is not linear. Assume $p$ is quadratic and there exists rational $b,c$ such that $x^2+bx+c$ has $\sqrt2+\sqrt 3$ as roots. Then on putting $x=\sqrt2+\sqrt3$ we get:
        \begin{align*}
            &(\sqrt 2+\sqrt3)^2+b(\sqrt2+\sqrt3)+c=0\\
            \Rightarrow& 5+c+b\sqrt2+(b+2\sqrt2)\sqrt3=0\\
            \Rightarrow&\sqrt3=-\frac{5+c+b\sqrt2}{b+2\sqrt2}\in\mathbb{Q}\sqrt2
        \end{align*}
        Which is not possible as shown in above. Therefore, $p$ is not quadratic. Now we have $deg(p)=[\mathbb Q(\sqrt2+\sqrt3):\mathbb Q]>2$. We have shown $[\mathbb Q(\sqrt2,\sqrt3):\mathbb Q]=4$. As $[\mathbb Q(\sqrt2+\sqrt3):\mathbb Q]\mid[\mathbb Q(\sqrt2,\sqrt3):\mathbb Q]$, the only possible value of $[\mathbb Q(\sqrt2+\sqrt3):\mathbb Q]$ is  4. It follows that $[\mathbb Q(\sqrt2+\sqrt3):\mathbb Q(\sqrt2,\sqrt3)]=[\mathbb Q(\sqrt2,\sqrt3):\mathbb Q]/[\mathbb Q(\sqrt2+\sqrt3):\mathbb Q]=1$. Therefore, $\mathbb{Q}(\sqrt2,\sqrt3)=\mathbb{Q}(\sqrt 2+\sqrt 3)$
    \end{proof}
    \begin{lemma}
        $\mathbb{Q}(\sqrt{2},\sqrt{3})$ is the splitting field of $(x^2-2)(x^2-3)$
    \end{lemma}
    \begin{proof}
        Note that $f(x)=(x^2-2)(x^2-3)=(x+\sqrt{2})(x-\sqrt{2})(x+\sqrt{3})(x-\sqrt{3})$. It is easy to see that the splitting field of $f$ contains $\sqrt{2},\sqrt{3}$ and the smallest such field is $\mathbb{Q}(\sqrt{2},\sqrt{3})$.
    \end{proof}\noindent
As $\mathbb{Q}(\sqrt{2}+\sqrt{3})=\mathbb{Q}(\sqrt{2},\sqrt{3})$ is the splitting field of $f$ over $\mathbb Q$, it is a normal extension of $\mathbb{Q}$.
\end{solution}







\begin{tcolorbox}[colback=c2,colframe=c1,title=Problem 2.4]
    Prove that the fields $\mathbb Q(\sqrt2)$ and $\mathbb Q(\sqrt3)$ are not isomorphic
\end{tcolorbox}
\begin{solution}
    Assume $\mathbb Q(\sqrt2)$ and $\mathbb Q(\sqrt3)$ are isomorphic and the isomorphism is given by $\phi$. Then $\phi(1)=1$ and for any rational $r\in\mathbb{Q}$, $\phi(r)=r\phi(1)=r$. Let $p(x)=\sum_{i=1}^nc_ix^i$ be the minimal polynomial of $\sqrt{2}$ over $\mathbb Q$. Then note that:
    $$\phi(p(x))=\phi\left(\sum_{i=1}^nc_ix^i\right)=\sum_{i=1}^n\phi(c_i)\phi(x)^i=\sum_{i=1}^nc_i\phi(x)^i=p\left(\phi(x)\right)$$ 
    Putting $x=\sqrt{2}$ we conclude that $\phi\left(\sqrt2\right)$ is also a solution of $p(x)$. We have shown in problem 1.1 that  $p(x)=x^2-2=(x-\sqrt{2})(x+\sqrt{2})$. Therefore, $\phi(\sqrt{2})\in\{\sqrt{2},-\sqrt{2}\}$. Assume $\sqrt2\in\mathbb Q(\sqrt3)$ and let $\sqrt 2=a+b\sqrt 3$ where $a,b\in\mathbb Q$. Then we note:
    \begin{align*}
        &\sqrt 2=a+b\sqrt 3\\
        \Rightarrow&2=a^2+3b^2+2ab\sqrt3\\
        \Rightarrow&\sqrt3=\frac{3-a^2-3b^2}{2ab}\in\mathbb{Q}
    \end{align*}
    Which is a contradiction. It follows that $\pm\sqrt{2}\notin \mathbb Q(\sqrt{3})$. Which contradicts the fact that $\phi(\sqrt{2})\in\mathbb{Q}(\sqrt{3})$. Therefore, no such $\phi$ exists. 
\end{solution}




\begin{tcolorbox}[colback=c2,colframe=c1,title=Problem 2.5]
    Let $f(X):=X^3+X^2+1\in \mathbb{Z}/2\mathbb{Z}[X]$ and $\alpha$ be a root of $f$. Show that $\mathbb{Z}/2\mathbb{Z}(\alpha)$ is the splitting field.
\end{tcolorbox}
\begin{solution}
    Note that $$0=f(\alpha)^2=(\alpha^3+\alpha^2+1)^2=\alpha^6+\alpha^4+1+2(\alpha^5+\alpha^3+\alpha^2)=\alpha^6+\alpha^4+1=f(\alpha^2)$$
    Therefore $\alpha^2$ is also a root(Note: We can easily check that 1,0 are not roots of the equation. Therefore, $\alpha\ne\alpha^2$). By using Vieta's formula, we get -1 is the product of the roots and thus $-1/\alpha^3$ is also a root. As the splitting field contains $\alpha$ and the smallest field containing $\alpha$ also contains all the other roots, we conclude that  $\mathbb{Z}/2\mathbb{Z}(\alpha)$ is the splitting field.
\end{solution}


\begin{tcolorbox}[colback=c2,colframe=c1,title=Problem 2.6]
    Let $p$ be a prime. Show that the splitting field of $X^{p}-  1 \in\mathbb Q[X]$ is of degree
    $p - 1$ over $\mathbb Q$.
\end{tcolorbox}
\begin{solution}
    Note if $\zeta$ is the $p^{th}$ root of unity, then all the roots are given by $\zeta^r$ where $0\leq r\leq p-1$. As the splitting field contains $\zeta$ and the smallest field containing $\zeta$ also contains all the other roots, we conclude that  $\mathbb Q(\zeta)$ is the splitting field. 
    Note that $X^p-1=(X-1)\left(\sum_{i=1}^{p-1}X^i\right)$. But As $\zeta\ne1$, $\left(\sum_{i=1}^{p-1}\zeta^i\right)=0$. We show $f(x)=\sum_{i=1}^{p-1}x^i$ is irreducible. Note that if $f(x)$ is irreducible if and only if $\tilde{f}(x)=f(x+1)$ is irreducible too (for if $f=hg$ then $\tilde{f}=h(x+1)g(x+1)$). Note that:
    $$\tilde{f}(x)=f(x+1)=\frac{(x+1)^p-1}{(x+1-1)}=\frac{(x+1)
    ^p-1}{x}=x^{p-1}+px^{p-1}\hdots {p\choose i}x^{i-1}\hdots +\frac{p(p-1)}{2}x+p$$ 
    We can apply Einstein's criteria on $\tilde{f}$ to conclude it is irreducible. Therefore $f$ is irreducible and is the minimal polynomial of $\zeta$. It follows that order of splitting field= degree of $f =p-1$.
\end{solution}


\begin{tcolorbox}[colback=c2,colframe=c1,title=Problem 2.7]
    Suppose $K | F$ and $L | K$ are normal extensions. Is $L | F$ normal?
\end{tcolorbox}
\begin{solution}
    Let $F=\mathbb{Q},K=\mathbb{Q}(\sqrt{2}),L=\mathbb{Q}(\sqrt[4]{2})$. Then $[K:F],[L:K]=2$ and as shown in 2.2, they are normal extension. Consider the polynomial $p(x)=x^4-2$. We  note $p$ is irreducible over $\mathbb{Q}$ by Einstein's criteria. Assume $L|F$ is normal. As $\sqrt[4]{2}$ is a root of $p$ then all roots of $p$ lies in $L$. Note $i\sqrt[4]{2}$ is a root of $p$ but is not in $L$, which is a contradiction. Therefore, $L|F$ is not normal and the statement is false.
\end{solution}



\begin{tcolorbox}[colback=c2,colframe=c1,title=Problem 2.8]
    Let $K_i
| F$ be finite extensions for $1 \leq i \leq n$. Then there exists a finite normal
extension $K | F$ and embeddings $\phi_i
: K_i\xrightarrow{\thicksim} K$ such that $\phi_i
|F = Id_F$.
\end{tcolorbox}
\begin{solution}
    Let $K_i$ as each $K_i$ is a finite extension, they are algebraic and are generated by adjoining a finite number of elements to $F$(one way to do so this is to adjoin elements one by one on $F$ till no elements remain. As $K_i$ has a finite order, this process stops in a finite number of steps). Let $K_i=F[\alpha_{(1,i)},\alpha_{(2,i)}\hdots]$. Consider the minimal polynomial $p_i=\prod_{i,j} p_{i,j}$ such that $p_{i,j}$ is the minimal polynomial of $\alpha_{i,j}$. Let $K$ be the splitting field of $p$ over $F$. As each $\alpha_{i,j}\in F$, $F$ satisfies all the conditions mentioned above.      
\end{solution}




\begin{tcolorbox}[colback=c2,colframe=c1,title=Problem 2.9]
    Let $p$ be a prime integer and $n \geq 1$. What is the splitting field of $x^{p^n}-1\in\mathbb{Z}/p\mathbb{Z}$?
\end{tcolorbox}
\begin{solution}
    We note that in $\mathbb{Z}/p\mathbb{Z}$ we have:
    $$x^{p^n}-1=x^{p^n}-1+\sum_{i=1}^{p-1}{p\choose i}\left(x^{p^{n-1}}\right)^i(-1)^{p-i}=(x^{p^{n-1}}-1)^p$$
    But in the same way we can write $(x^{p^{n-1}}-1)=(x^{p^{n-2}}-1)^p$ and so on. Continuing, we get $(x^{p^{n}}-1)=(x-1)^{p^n}$ in $\mathbb Z/p\mathbb{Z}$. Therefore, the only root is 1, and the splitting field is $\mathbb{Z}/p\mathbb{Z}$.
\end{solution}





\begin{tcolorbox}[colback=c2,colframe=c1,title=Problem 2.10]
    Let $K | F$ be a finite normal extension and $f(x) \in F[x]$ be irreducible. Let $g, h \in K[x]$ be two irreducible factors of $f$ in $K[x]$. Show that there exists an $F$-isomorphism $\sigma : K\xrightarrow[]{\sim} K$ that  takes $g$ to $h$
\end{tcolorbox}


\section{Problem sheet 3}

\begin{tcolorbox}[colback=c2,colframe=c1,title=Problem 3.1]
    Check whether $X^4 + X + 1 \in\mathbb Q[X]$ is a separable polynomial.
\end{tcolorbox}
\begin{solution}
    $Df=4X^3+1$. Note that
    $$f=\frac{x}{4}Df+\left(\frac{3x}{4}+1\right)$$
    Assume $f$ is inseparable and they share a solution $\alpha$. Put $x=\alpha$ above to get 
    $$\left(\frac{3}{4}\alpha+1\right)=f(\alpha)-\frac{\alpha}{4}Df(\alpha)=0\Rightarrow \alpha=-\frac{4}{3}$$
    It is easy to check $f\left(-\frac{4}{3}\right)\ne0$ which contradicts the fact that $\alpha$ is a solution. Therefore, no such $\alpha$ exists and $f$ is separable. 
\end{solution}



\begin{tcolorbox}[colback=c2,colframe=c1,title=Problem 3.2]
    Show that any field of characteristic 0 is perfect.
\end{tcolorbox}
\begin{solution}
    Let $F$ be a characteristic 0 field and let $f=\sum_{i=0}^nc_ix^i$ be irreducible. Let $\alpha$ be a root of the polynomial. Then $f$ is the minimal polynomial of $\alpha$ over $F$. We claim that multiplicity of $\alpha$ in $f$ is 1. If $f$ has degree 1, then it has a single root $\alpha$ and we are done. Otherwise, we have $Df(\alpha)=0$. Now, if we assume $f$ has degree $n$ then $Df =\sum_{i=0}^{n-1}(i+1)c_{i+1}x^i$. As $c_n\ne 0$, and $char(F)=0$, $nc_n\ne0$. Therefore, $Df\ne0$. But as $Df(\alpha)=0$ and  $deg(Df)=deg(f)-1$, $p$ cannot be the minimal polynomial, which is a contradiction. Therefore, $Df(\alpha)\ne0$ and multiplicity of $\alpha$ in $f$ is 1. So $F$ is perfect.
\end{solution}


\begin{tcolorbox}[colback=c2,colframe=c1,title=Problem 3.3]
    Let $F$ be a field of characteristic $p$. Show that the map $\phi : F \to F$ defined by
$\phi(a) = a^p$
is a homomorphism.
\end{tcolorbox}
\begin{solution}
    We know for any prime $p$(characteristic of a field is always prime) and $1<k<p$, $p\choose k$ is divisible by $p$. 
    For $a,b\in F$
    $$\phi(a)+\phi(b)=a^{p}+b^p=a^{p}-1+\sum_{i=1}^{p-1}{p\choose i}\left(a\right)^i(b)^{p-i}=(a+b)^p=\phi(a+b)$$
    As
    \begin{enumerate}
        \item $\phi(a+b)=\phi(a)+\phi(b)$
        \item $\phi(ab)=(ab)^p=a^pb^p=\phi(a)\phi(b)$
    \end{enumerate}
    $\phi$ is a homomorphism.
\end{solution}

\begin{tcolorbox}[colback=c2,colframe=c1,title=Problem 3.4]
    Show that any algebraic extension of a perfect field is perfect.
\end{tcolorbox}
\begin{solution}
    
\end{solution}






\begin{tcolorbox}[colback=c2,colframe=c1,title=Problem 3.5]
    A nonzero polynomial $f(x) \in F[x]
    $ is separable if and only if it is relatively
prime to its derivative in $F[x]$ (i.e., $(f(x), f'
(x)) = 1))$.
\end{tcolorbox}
\begin{solution}\\
    \textbf{$f$ is separable $\Rightarrow(f,f')=1$}\\
    Assume to the contrary that  $(f,f')=p\ne 1$. Then all roots of $p$ is a common root of both $f$ and $f'$. Therefore, $f$ is not separable which is a contradiction.\\

    \noindent\textbf{$(f,f')=1\Rightarrow$ $f$ is separable}\\
    Assume $f$ and $f'$ are inseparable. Then there exists $\alpha$ such that $f(\alpha)=f'(\alpha)=0$. Let $p$ be the minimal polynomial of $\alpha$. Then $p\mid f,p\mid f'$ and so $p\mid (f,f')$. But this is a contradiction as $1$ is a scalar and non-scalar polynomial can divide it. 
\end{solution}








\begin{tcolorbox}[colback=c2,colframe=c1,title=Problem 3.6]
    Is $f(X) = X^6+X^5+X^4+2X^3+2X^2+X +2 \in\mathbb F_3[X]$ a separable polynomial?
\end{tcolorbox}
\begin{solution}
    In $\mathbb{F}_3$, $f'=2x^4++x^3+x$. We use Euclid's algorithm to compute the GCD.
    \begin{align*}
        x^6+x^5+x^4+2x^3+2x^2+x+2=(2x^2+x)(2x^4+x^3+x)+(x^2+x+2)
    \end{align*}
    \begin{align*}
        2x^4+x^3+x=(x^2+x+2)(2x^2+2x)+x
    \end{align*}
    \begin{align*}
        x^2+x+2=x(x+1)+2
    \end{align*}
    Therefore, $(f,f')=1$ and by problem 3.5, $f$ is separable.
\end{solution}




\section{Problem Sheet 4}

\begin{tcolorbox}[colback=c2,colframe=c1,title=Problem 4.1] 
    Let $F$ be a finite field with $p^n$
    elements. Show that the map $\phi : F \to F$ defined
    by $\phi(a) = a^p$    is an isomorphism. Show that $\phi$ has order $n$.    
\end{tcolorbox}
\begin{solution}
    As $F$ is finite the characteristic of $F$ is a prime. Let it be $k$. Then for $a\in K$, $ka=0$. As $(F,+)$ forms a group, by Lagrange's theorem, $k|p^n$. The only prime which divides $p^n$ is $p$, so $k=p$. By problem 3.3, $\phi$ is a group homomorphism. Let $\phi(\alpha)=0$. Then $\alpha^p=0\Rightarrow \alpha=0$. Therefore, $\ker(\phi)=\{0\}$ and $\phi$ is one-one. As $F$ is finite, any injective map from $F$ to itself is a bijection. Therefore, $\phi$ is an isomorphism.
\end{solution}



\begin{tcolorbox}[colback=c2,colframe=c1,title=Problem 4.2]
    Show that if $m$ divides $n$, $x^{p^m}-x$ divides $x^{p^n}-x$
\end{tcolorbox}
\begin{solution}
    \begin{lemma}
        If $m|n$ then $x^m-1|x^n-1$.
    \end{lemma}
    \begin{proof}
        Write $n=mk$. Then $$x^n-1={x^m}^k-1=(x^m-1)\left(\sum_{i=0}^{k-1}x^m\right)$$
    \end{proof}
    Now as $m|n$, $$p^m-1|p^n-1\Rightarrow x^{p^m-1}-1|x^{p^n-1}-1\Rightarrow x(x^{p^m-1}-1)|x(x^{p^n-1}-1)\Rightarrow x^{p^m}-x|x^{p^n}-x$$
\end{solution}

\begin{tcolorbox}[colback=c2,colframe=c1,title=Problem 4.3]
    Investigate whether there is a finite field with the following number of
elements and construct such a field if it exists.\\
(1) 72.\\
(2) 625.\\
\end{tcolorbox}
\begin{solution}
    \begin{enumerate}
        \item No. Finite fields always have $p^n$ elements where $p$ is a prime. 72 is not of this form.
        \item Yes. $625=5^4$. As $5$ is a prime, such a field exists. It is the splitting field of $x^{625}-x$ over $\mathbb F_5$
    \end{enumerate}
\end{solution}



\begin{tcolorbox}[colback=c2,colframe=c1,title=Problem 4.4]
    Let F be a field with $ch(F) = p$. Give an example to show that the map
$\phi: F \to F$ defined by $\phi(a) = a^p$ need not be an automorphism of $F$ if $F$ is an
infinite field.
\end{tcolorbox}
\begin{solution}
    Let $p=3$ and $F=\mathbb{F}_3[x]$. If $\phi$ is an automorphism, then there exists $f\in\mathbb F_3[x]$ such that $\phi(f)=f^3=x^3+x$ and $deg(f^3)=3deg(f)=3\Rightarrow deg(f)=1$. Let $f=x-a$. Then $f^3=x^3-a^3$, which cannot be equal to $x^3+x$.  Therefore, no such $f$ exists and $\phi$ is not an automorphism.
\end{solution}



\begin{tcolorbox}[colback=c2,colframe=c1,title=Problem 4.5]
    Let $K | F$ be a finite extension where $F$ is a finite field. Show that $|K| = (|F|)^n$ for some $n \in\mathbb N$.
\end{tcolorbox}
\begin{solution}
    As $F,K$ are finite fields, $|K|=p^\alpha$ for some primes $p$. As $F$ is a subfield, characteristic of $F$ is also $p$ and $|F|=p^\beta$ where $\beta<\alpha$. We know $\beta|\alpha$ and therefore there exists $c$ such that $\beta c=\alpha\Rightarrow (p^\beta)^c=p^\alpha\Rightarrow |F|^c=|K|$
\end{solution}



\begin{tcolorbox}[colback=c2,colframe=c1,title=Problem 4.6]
    If $f(x) \in F[x]$ is separable then the splitting field of $f(x)$ over $F$ is separable
    over $F$.
\end{tcolorbox}
\begin{solution}
    
\end{solution}




\begin{tcolorbox}[colback=c2,colframe=c1,title=Problem 4.7]
    Show that $\mathbb Q(\sqrt 2,\sqrt[3]3)$ is simple by showing $\mathbb Q(\sqrt 2+\sqrt[3]3)=\mathbb Q(\sqrt 2,\sqrt[3]3)$
\end{tcolorbox}
\begin{solution}
   Note that $\mathbb{Q}\hookrightarrow\mathbb Q(\sqrt 2+\sqrt[3]3)\hookrightarrow\mathbb Q(\sqrt 2,\sqrt[3]3)$. By following the steps similar to problem 1.1, we get $\mathbb Q(\sqrt 2,\sqrt[3]3)=6$. Let $p$ be the minimal polynomial of $\sqrt2+\sqrt[3]3$ over $\mathbb{Q}$. $p$ is not linear. We show $p$ is not quadratic or cubic. Assume $p$ is quadratic. Then $p=x^2+bx+c$ for some $b,c\in\mathbb Q$. Therefore:
   \begin{align*}
    &(\sqrt2+\sqrt[3]3)^2+b(\sqrt2+\sqrt[3]3)+c=0\\
    \Rightarrow& \sqrt{2}=\frac{-c-b\sqrt[3]{3}-\sqrt[3]{3}^2-2}{b+2\sqrt[3]{3}}\in\mathbb{Q}(\sqrt[3]{3})
   \end{align*}
   Which is not true. A similar argument shows $p$ is not cubic. Therefore $deg(p)>3$ and by tower lemma, $deg(p)=[\mathbb Q(\sqrt 2+\sqrt[3]3):\mathbb{Q}]|6$. Therefore $[\mathbb Q(\sqrt 2+\sqrt[3]3):\mathbb{Q}]=6$ and $[\mathbb Q(\sqrt 2,\sqrt[3]3):\mathbb{Q}]|[\mathbb Q(\sqrt 2+\sqrt[3]3):\mathbb{Q}]=1\Rightarrow [\mathbb Q(\sqrt 2+\sqrt[3]3):\mathbb{Q}]=[\mathbb Q(\sqrt 2,\sqrt[3]3):\mathbb{Q}]$
\end{solution}


\begin{tcolorbox}[colback=c2,colframe=c1,title=Problem 4.8]
    Prove that every element of a finite field can be written as a sum of two
    squares.
\end{tcolorbox}
\begin{solution}
    \begin{lemma}
        For $a\in F,a\ne 0$, if $a$ has a square root, then $a$ has exactly 2 square roots. 
    \end{lemma}
    \begin{proof}
        Let $x^2=a$ and let $(x+y)^2=a$. Then it follows that $2xy+y^2=0\Rightarrow y(2x+y)=0$. Therefore, $y=0$ or $y=-2x$. Therefore, there are exactly two roots: $x,-x$. For $a=0,x=-x=0$
    \end{proof}\noindent
    It follows that $S=\{a|\exists$ $x$ such that $x^2=a\}$ has exactly $\frac{|F|-1}{2}+1=\frac{|F|+1}{2}$ elements. For any $a\in F$ consider $S_a=\{a-s|s\in S\}$. Then $S_a$ also has  $\frac{|F|+1}{2}$ elements. As $|S|+|S_a|=|F|+1>|F|$ by pigeonhole principle, $S\cap S_a\ne\phi$. Let $e$ be an element in the intersection. Then $e=n^2=a-m^2$ for $n,m\in F$. Therefore, $a=n^2+m^2$. 
\end{solution}





\noindent\rule{7in}{2.8pt}

\end{document}
 