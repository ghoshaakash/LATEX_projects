\documentclass{article}
\usepackage[a4paper, total={7in, 9.5in}]{geometry}
\usepackage[utf8]{inputenc}
\usepackage[T1]{fontenc}
\usepackage{titlesec, blindtext, color}
\usepackage{amsmath}
\usepackage{amsfonts}
\usepackage{graphicx}
\usepackage{adjustbox}
\usepackage{booktabs}
\usepackage{hyperref}
\usepackage{changepage}
\usepackage{float}
\usepackage{ amssymb }
\usepackage{adjustbox}
\usepackage[english]{babel}
\usepackage{ mathdots }
\usepackage{tcolorbox}
\usepackage{pdfpages}
\usepackage{booktabs}
\usepackage{amsthm}

\newtheorem{problem}{Problem}
\newtheorem{lemma}{Lemma}
\setlength{\parindent}{0pt}

\DeclareMathOperator{\Int}{Int}
\DeclareMathOperator{\Bd}{Bd}


\title{Topology, Munkres\\ Chapter 4}
\author{Aakash Ghosh}
\date{March 2021}

\begin{document}
\maketitle
\begin{center}
    \Large{\textbf{Exercise Set 1}}
\end{center}
\begin{tcolorbox}
\begin{problem}
\begin{enumerate}
    \item A $G_\delta$ set in a space $X$ is a set $A$ that equals a countable intersection of open sets of $X$. Show that in a first-countable $T_1$ space, every one-point set is a $G_\delta$ set.
    \item There is a familiar space in which every one-point set is a $G_\delta$ set, which nevertheless does not satisfy the first countability axiom. What is it?
\end{enumerate}
The terminology here comes from the German. The “$G$” stands for “Gebiet,”
which means “open set,” and the “$\delta$” for “Durchschnitt,” which means “intersection.
\end{problem}
\end{tcolorbox}
\textbf{Solution :}











\end{document}
