\documentclass[oneside]{book}
\usepackage[utf8]{inputenc}
\usepackage{pgfplots}
\pgfplotsset{compat=1.15}
\usepackage{mathrsfs}
\usetikzlibrary{arrows}
\usepackage{amsmath}
\usepackage{tikz}
\usepackage{wrapfig}
\usepackage{parskip}
\usepackage[most]{tcolorbox}
\usepackage{amssymb}
\usepackage{amsthm}
\usepackage[margin=0.7in]{geometry}
\usepackage{mathtools}
\usepackage{caption}
\usepackage{float}
\usepackage{enumitem}

\newtheorem{theorem}{Theorem}
\newtheorem{definition}[theorem]{Definition}
\newtheorem{claim}[theorem]{Claim}
\newtheorem{proposition}[theorem]{Proposition}
\newtheorem{lemma}[theorem]{Lemma}
\newtheorem{corollary}[theorem]{Corollary}
\newtheorem{conjecture}[theorem]{Conjecture}
\newtheorem*{observation}{Observation}
\newtheorem*{example}{Example}
\newtheorem*{remark}{Remark}

\usepackage{physics}
\usepackage{amsmath}
\usepackage{tikz}
\usepackage{mathdots}
\usepackage{yhmath}
\usepackage{cancel}
\usepackage{color}
\usepackage{siunitx}
\usepackage{array}
\usepackage{multirow}
\usepackage{amssymb}
\usepackage{gensymb}
\usepackage{tabularx}
\usepackage{extarrows}
\usepackage{booktabs}
\usetikzlibrary{fadings}
\usetikzlibrary{patterns}
\usetikzlibrary{shadows.blur}
\usetikzlibrary{shapes}




\tcbset{mytitle/.style={title={Question~\thetcbcounter\ifstrempty{#1}{}{: #1}}}}
\newtcolorbox[auto counter, number within=chapter, number freestyle={\noexpand\thechapter.\noexpand\arabic{\tcbcounter}}]{question}[1][]{%
    enhanced,
    breakable,
    fonttitle=\bfseries,
    mytitle={},
    #1
}


\title{Solutions to Analytic Number Theory by Apostol}
\author{Aakash Ghosh }
\begin{document}
\maketitle
\tableofcontents

\chapter{The Fundamental Theorem of Arithmetic}
\textbf{In these exercises lower case latin letters $a, b, c, ... , x, y, z$ represent integers. 
Prove each of the statements in Exercises I through 6.}\\\\
\begin{tcolorbox}
\textbf{1.1. lf $(a, b)= 1$ and if $c|a$ and $d|b$, then $(c, d)= 1$.}
\end{tcolorbox}
\textbf{Solution:} Assume to the contrary that $(c,d)=m\ne 1$. Write $c=k_cm,d=k_dm$ Then $a=ck_a=k_ak_cm$, $b=k_bk_dm$. Then $(a,b)$ is at least $m$, which is a contradiction.\\\\
\textbf{Alt Solution:}We can write $a=k_1c$ and $b=k_2d$. Then as $(a,b)=1$ there exists $x,y$ such that $ax+by=1\Rightarrow c(k_1x)+d(k_2y)=1$ which implies $(c,d)=1$. 


\begin{tcolorbox}
\textbf{1.2. If $(a, b)= (a, c)= 1$, then $(a, bc)= 1$. }
\end{tcolorbox}
\textbf{Solution :}Assume to the contary that $(a,bc)\ne 1$. Then there exist prime $p$ which divides $(a,bc)$ therefore, $p|a$ and $p|bc$. But if $p|bc$ then $p|b$ or $p|c$. But either case will lead to contradiction as this implies either $b$ or $c$ share a common factor with $a$ (which is $p$).\\\\
\textbf{Alt Solution:} There exist $x_1y,y_1$ and $x_2,y_2$ such that:
$$ax_1+by_1=1\quad ax_2+cy_2=1$$
Multiply them to get:
$$a(ax_1x_2+bx_2y_1+cx_1y_2)+bc(y_1y_2)=1$$
Which implies $(a,bc)=1$

\begin{tcolorbox}
\textbf{1.3. lf $(a, b)= 1$, then $(a^m, b^n) = 1$ for all $n,k\geq 1$ }
\end{tcolorbox}
\textbf{Solution :}Assume to the contary that $(a^m,b^n)\ne 1$. Then there exist prime $p$ which divides $(a^n,b^m)$ therefore, $p|a^n$ and $p|b^m$. As $p|b^m$ then $p|b$. Similarly, $p|a$. This leads to contradiction as this implies either $a$ and $b$ share a common factor (which is $p$).\
\begin{tcolorbox}
\textbf{1.4. If $(a, b)= 1$, then $(a+ b, a-b)$ is either 1 or 2.}
\end{tcolorbox}
\textbf{Solution :}As $(a,b)=1$, there exist $(x,y)$ such that $ax+by=1$. Then:
\begin{align*}
    &ax+by=1\\
    \Rightarrow&\frac{(a+b)+(a-b)}{2}x+\frac{(a+b)-(a-b)}{2}y=1\\
    \Rightarrow&(a+b)\frac{(x+y)}{2}+(a-b)\frac{(x-y)}{2}=1\\
    \Rightarrow&(a+b)(x+y)+(a-b)(x-y)=2
\end{align*}
Note if $a'x'+b'y'=m$ then $(a',b')|m$, Therefore, $(a+b,a-b)|2$ or $(a+b,a-b)$ is 1 or 2.

\begin{tcolorbox}
\textbf{1.5 If $(a, b) = 1$, then $(a + b, a^2 - ab + b^2 )$ is either 1 or 3.} 
\end{tcolorbox}




\begin{tcolorbox}
\textbf{1.6. If $(a, b)= 1$ and if $d|(a +b)$, then $(a, d)= (b, d)= 1$. }
\end{tcolorbox}
\textbf{Solution :}As $(1,b)=1$, there exists $x,y$ such that $ax+by=1$. Write $a+b=dk$. Then:
$$ax+by=1\Rightarrow a(x-y)+(a+b)y=1\Rightarrow a(x-y)+d(ky)=1$$
Therefore, $(a,d)=1$. Replace $a$ with $b$ everywhere above to get $(b,d)=1$.


\begin{tcolorbox}
\textbf{1.7. A rational number $a/b$ with $(a, b) = 1$ is called a reduced fraction. If the sum of two 
reduced fractions is an integer, say $(a/b)+ (c/d) = n$, prove that $|b|=|d|$.}
\end{tcolorbox}
\textbf{Solution :}By altering sign of $a,c$ we can keep $b,d>0$. We have: $ad+bc=nbd\Rightarrow ad=b(nd-c)$, Now, as $(a,b)=1$, $b|d$. Similarly, $d|b$ and thus $d=\pm b$ and $|d|=|b|$



\begin{tcolorbox}
\textbf{1.8 An integer is called \textit{squarefree} if it is not divisible by the square of any prime. Prove 
that for every $n \geq 1$ there exist uniquely determined $a > 0$ and $b > 0$ such that 
$n = a^2b$, where $b$ is square free. } 
\end{tcolorbox}
\textbf{Solution :}
Let $n=\prod p_i{\alpha_i}$ where each $p_i$ is prime. We write $\alpha_i=2\beta_i+r_i$ where $\beta_i>0$ and $0\leq r_i<2$. Then set $a=\prod p_i^{\beta_i}$ and $b=\prod p_i^{r_i}$. It follows $b$ is square free as if $p_i^2|b$ then $r_i\geq2$ which leads to a contradiction. Being unique follows from construction.


\begin{tcolorbox}
\textbf{1.9. For each of the following statements, either give a proof or exhibit a counter example. \begin{enumerate}
    \item If $b^2|n$ and $a^2|n$ and $a^2<b^2$ then $a|b$
    \item If $b^2$ is the largest square divisor of n, then $a^2|n$  implies $a | b$.
\end{enumerate}}
\end{tcolorbox}
\textbf{Solutions:}
\begin{enumerate}
    \item No. Set $n=36$, $a=2$, $b=3$.
    \item Yes. If $n=\prod p_i^{\alpha_i}$ then $b=\prod p_i^{\beta_i}$ where $\alpha_i$ ,$p_i$ and $\beta_i$ are as defined in above problem. If $a^2|n$ and if $a=\prod p_i^{a_i}$ then $2a_i\leq \alpha_i\Rightarrow a_i<\beta_i$. Therefore, $a_i|\beta_i$ and $a|b$.
\end{enumerate}

\begin{tcolorbox}
\textbf{1.10. Given $x$ and $y$, let $m = ax+ by$, $n =ex + dy$, where $ad- be = \pm1$. Prove that 
$(m, n) = (x, y)$. }
\end{tcolorbox}
\textbf{Solution: }By the equations given, $(a,b)|m$ and $(a,b)|n$. Therefore, $(a,b)|(m,n)$. Now $md-nb=(ad-be)x=\pm x$.  So, $((m,n)|x$ and in a similar way we get, $(m,n)|y$. So $(m,n)|(x,y)$ and $(m,n)=(x,y)$.

\begin{tcolorbox}
\textbf{1.11. Prove that $n^4 + 4$ is composite if $n > 1$.}
\end{tcolorbox}
\textbf{Solution :}Note:
$$n^2+4=n^4+4n^2+4-4n^2= (n^2+2)^2-4n^2=(n^2-2n+2)(n^2+2n+2)$$
For $n>1$
, the quardratics are  posative, and so being composite follows.\\\\
\textbf{In Exercises 12, 13, and 14, $a, b, c, m, n$ denote positive integers.}\\\\
\begin{tcolorbox}
\textbf{1.12. For each of the following statements either give a proof or exhibit a counter example. 
\begin{enumerate}
    \item If $a^n|b^n$ then $a|b$.
    \item $n^n|m^m$, then $n|m$.
    \item If $a^n|2b^n$ and $n>1$ then $a|b$.
\end{enumerate}
}
\end{tcolorbox}
\textbf{Solution :}
\begin{enumerate}
    \item Let $a=\prod p_i^{a_i}$ and $b=\prod p_i^{b_i}$. Then $a^n|b^n\Rightarrow na_i|nb_i\Rightarrow a_i|b_i\Rightarrow a|b$
    \item 
\end{enumerate}






\chapter{}

\end{document}
