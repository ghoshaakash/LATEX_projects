\documentclass[oneside]{article}
\usepackage[utf8]{inputenc}
\usepackage{pgfplots}
\pgfplotsset{compat=1.15}
\usepackage{mathrsfs}
\usetikzlibrary{arrows}
\usepackage{amsmath}
\usepackage{tikz}
\usepackage{wrapfig}
\usepackage{parskip}
\usepackage[most]{tcolorbox}
\usepackage{amssymb}
\usepackage{amsthm}
\usepackage[margin=0.7in]{geometry}
\usepackage{mathtools}
\usepackage{caption}
\usepackage{tfrupee}
\usepackage{booktabs}
\usepackage{float}


\let\orupee\rupee
\def\rupee{\ifmmode\text{\orupee}\else\orupee\fi}


\newtheorem{theorem}{Theorem}
\newtheorem{definition}[theorem]{Definition}
\newtheorem{claim}[theorem]{Claim}
\newtheorem{proposition}[theorem]{Proposition}
\newtheorem{lemma}[theorem]{Lemma}
\newtheorem{corollary}[theorem]{Corollary}
\newtheorem{conjecture}[theorem]{Conjecture}
\newtheorem*{observation}{Observation}
\newtheorem*{example}{Example}
\newtheorem*{remark}{Remark}

\usepackage{physics}
\usepackage{amsmath}
\usepackage{tikz}
\usepackage{mathdots}
\usepackage{yhmath}
\usepackage{cancel}
\usepackage{color}
\usepackage{siunitx}
\usepackage{array}
\usepackage{multirow}
\usepackage{amssymb}
\usepackage{gensymb}
\usepackage{tabularx}
\usepackage{extarrows}
\usepackage{booktabs}
\usetikzlibrary{fadings}
\usetikzlibrary{patterns}
\usetikzlibrary{shadows.blur}
\usetikzlibrary{shapes}




\tcbset{mytitle/.style={title={Question~\thetcbcounter\ifstrempty{#1}{}{: #1}}}}
\newtcolorbox[auto counter, number within=chapter, number freestyle={\noexpand\thechapter.\noexpand\arabic{\tcbcounter}}]{question}[1][]{%
    enhanced,
    breakable,
    fonttitle=\bfseries,
    mytitle={},
    #1
}


\title{Outer Measure and Related Things}
\author{Aakash Ghosh }
\date{Instructor: Prof. Shirshendu Choudhury}
\begin{document}
\maketitle





\section{Outer Measure}
For a given $X$, a outer measure $\mu *: \mathcal{P}(X)\to(0,\infty]$ is defined to be function with the following properties:
\begin{enumerate}
    \item $\mu^* (\phi)=0$
    \item If $A\subseteq B$ then $\mu^*(A)\leq \mu^*(B)$
    \item $\mu^*\left(\bigcup_{i=1}^\infty A_i\right)\leq \sum_{i=1}^\infty \mu^*(A_i)$ $\forall A_i\in\mathcal{P}(X)$.
\end{enumerate}

\textbf{Example:} Let $\epsilon\subseteq\mathcal P(X)$ and $f:\epsilon:[0,\infty)$ such that:
\begin{enumerate}
    \item $\phi,X\in \epsilon$
    \item $f(\phi)=0$
\end{enumerate}
Then for a subset $A\subseteq X$ define: 
$$\mu^*(A)=\inf\left\{\left(\sum_{i=0}^\infty\mu^*(A_i)\right)|A\subseteq \bigcup_{i=1}^\infty A_i\right\}$$
We show that $\mu^*$ defined above is an outer measure.
\begin{enumerate}
    \item $\mu^*(\phi)=0$ as $\phi\subseteq\phi$
    \item Let $A\subseteq B$. As all covers of $B$ are also covers of $A$, $\mu^*(A)<\mu^*(B)$.
    \item Let $\cup_{j=1}^\infty E_i^j$ be a cover of $A_i$ in $\epsilon$. Then, $\bigcup_{i,j}E_i^j$ covers $A$. Therefore,  $\mu *\left(\bigcup_{i=1}^\infty A_i\right)\leq \sum_{i=1}^\infty \mu^*(A_i)$ $\forall A_i\in\mathcal{P}(X)$
\end{enumerate}
In particular, we can set $X=\mathbb{R}$, $\epsilon=\{(a,b)|a,b\in[-\infty,\infty]\}$, and $f(I)=$ length if interval $I$. 
\begin{definition}[$\mu^*$ measurable sets]
    A set $A\subseteq X$ is called $\mu^*$ measurable if for all $E\in\mathcal{P}(X)$ we have:
    $$\mu^*(E)=\mu^*(E\cap A)+\mu^*(E\cap A^c)$$    
\end{definition}
Note, as $E\cap A$ and $E\cap A^c$ covers $A$, then it is often enought to prove $\mu^*(E)\geq\mu^*(E\cap A)+\mu^*(E\cap A^c)$. The other way follows trivially from the definition of $\mu^*$

\section{Caratheodory's Extension Theorem}
THis theorem is used to generates mesure from outer measures. 
\begin{definition}[Complete Measure]
    Let $\mu$ be a measure. Let $A$ have size zero and $B\subseteq A$. $\mu$ is a complete measure if $B$ is measurable and $\mu(B)=0$.
\end{definition}

\begin{theorem}
    Let $X$ be a non empty set with outer measure $\mu^*$. Let $\mathcal{M}$ be the set of all $\mu^*$ measurable subset of $X$. Let $\mu=\mu^*|_\mathcal{M}$. Then $(X,\mathcal{M},\mu)$ forms a complete measure.
\end{theorem}

\begin{proof}
    Put $A=\phi$ and $A=\phi$ in place of $A$ in the statement of $\mu^*$ measurable function to get $X,\phi\in\mathcal{M}$. If $A\in\mathcal{M}$ then $\forall E\in\mathcal{P}(X)$, $\mu^*(E)=\mu^*(E\cap A)+\mu^*(E\cap A^c)=\mu^*(E\cap A^c)+\mu^*(E\cap (A^c)^c)$. Therefore $A^c\in\mathcal M$.\\
    For showing $\mathcal{M}$ is closed under taking countable union, we shall first prove it for the finite case and then take a limit under apropiate conditions.\\
    \textbf{Showing $\mathcal{M}$ is closed under finite union}: We just need to show that it holds for union of two sets and then the proof for the union of any arbitary $n<\infty$ set follows from induction. 
    \begin{align}
        \mu^*(E)&=\mu^*(E\cap A)+\mu^*(E\cap A^c)\\
        &=\mu^*(E\cap A\cap B)+\mu^*(E\cap A\cap B^c)+\mu^*(E\cap A^c\cap B)+\mu^*(E\cap A^c\cap B^c)
    \end{align}
    We go from (1) to (2) by using the $\mu^*$ measurability of $B$ for $E\cap A$ and $E\cap A^c$.  We know: $A\cup B=(A\cap B)\cup(A^c\cap B)\cup(A\cap B^c)$. Therefore:
    \begin{align}
        \mu^*(E)&=\mu^*(E\cap A\cap B)+\mu^*(E\cap A\cap B^c)+\mu^*(E\cap A^c\cap B)+\mu^*(E\cap A^c\cap B^c)\\
        &=\geq \mu^*(E\cap(A\cup B))+\mu^*(E\cap A^c\cap B^c)\\
        &=\geq \mu^*(E\cap(A\cup B))+\mu^*(E\cap (A\cup B)^c)
    \end{align}  
    As we have mentioned before, this is enough to conclude $A\cup B$ is $\mu^*$ measurable. Note:
    \begin{align}
        \mu^*(A\cup B)=\mu^*(A\cup B\cap A)+\mu^*(A\cup B\cap A^c)=\mu^*(A)+\mu^*(B)
    \end{align}
    As $\mu^*$ and $\mu$ agree on measurable sets, the result for finite unions is proved.\\
    Now we sahll attempt the case for the union of infinite number of sets. 
    Define:
    $$B_n=\bigcup_{i=1}^n A_i\quad B=\bigcup_{i=1}^\infty A_i$$
    where each $A_i\in\mathcal{M}$. Therefore:
    \begin{align}
        \mu^*(E\cap B_n)=\mu^*(E\cap B_n\cap A_n)+\mu^*(E\cap B_n\cap A_n^c)=\mu^*(E\cap A_n)+\mu^*(E\cap B_{n-1})
    \end{align}
    As this is a recurrence relation we get:
    \begin{align}
        \mu^*(E\cap B_n)=\sum_{i=1}^n\mu^*(E\cap A_n)    
    \end{align}
    Therefore:
    \begin{align}
        \mu^*(E)&=\mu^*(E\cap B_n)+\mu^*(E\cap B_n^c)\\
        &\geq \sum_{i=1}^n\mu^*(E\cap A_i)+\mu^*(E\cap B_n^c)
    \end{align}
    Now we take the limit $n\to\infty$ to get:
    \begin{align}
        \mu^*(E)&\geq \sum_{i=1}^\infty\mu^*(E\cap A_i)+\mu^*(E\cap B_n^c)\\
        &\geq \mu^*(E\cap \bigcup_{i=1}^\infty(A_i)+\mu^*(E\cap B_n^c)\\
        &\geq\mu^*(E\cap B)+\mu^*(E\cap B^c)
    \end{align}
    Therefore, $\mathcal{M}$ is closed under countable unions. Take $B=E$ in (11) to get
    $\mu^*(B)\geq \sum_{i=1}^\infty \mu^*(B\cap A_i)=\sum_{i=1}^\infty \mu^*(A_i)$. As by definition, $\mu^*(B)\leq \sum_{i=1}^\infty \mu^*(A_i)$, equality holds. Therefore, $\mu$ is a measure. Now we need to show this measure is complete. Let $\mu^*(A)=0$ and $B\subseteq A$. Then:
    $$\mu^*(E)\geq \mu^*(A\cap E)+\mu^*(A^c\cap E)\geq \mu^*(B\cap E)+\mu^*(B^c\cap E)\geq \mu^*(E)$$
    Therefore, equality must hold and $\mu^*(B)$ is measurable. This completes the proof.
\end{proof}
\section{Leabesgue Measure}
We perform Caratheodory extension on the outer measure on $\mathbb R$ defined in the example. The measure we get is known as the leabesgue measure and is written as $(\mathbb{R},\mathcal{M}_1,m_i)$. The notion can be generalised for higher dimensions where we take $(a_1,b_1)\times (a_2,b_2)\hdots $ to be elements of $\epsilon$ and $f((a_1,b_1)\times (a_2,b_2))=$Volume/Area of the figure. 
\end{document}
