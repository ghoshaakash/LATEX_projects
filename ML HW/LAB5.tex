\documentclass[a4paper, 11pt]{article}
\usepackage{comment} % enables the use of multi-line comments (\ifx \fi) 
\usepackage{lipsum} %This package just generates Lorem Ipsum filler text. 
\usepackage[a4paper, total={7in, 10in}]{geometry}
\usepackage{amsmath}
\usepackage{amssymb,amsthm}  % assumes amsmath package installed
\newtheorem{theorem}{Theorem}
\newtheorem{corollary}{Corollary}
\usepackage{graphicx}
\usepackage{tikz}
\usetikzlibrary{arrows}
\usepackage{verbatim}
\usepackage[numbered]{mcode}
\usepackage{float}
\usepackage{tikz}
    \usetikzlibrary{shapes,arrows}
    \usetikzlibrary{arrows,calc,positioning}

    \tikzset{
        block/.style = {draw, rectangle,
            minimum height=1cm,
            minimum width=1.5cm},
        input/.style = {coordinate,node distance=1cm},
        output/.style = {coordinate,node distance=4cm},
        arrow/.style={draw, -latex,node distance=2cm},
        pinstyle/.style = {pin edge={latex-, black,node distance=2cm}},
        sum/.style = {draw, circle, node distance=1cm},
    }
\usepackage{xcolor}
\usepackage{mdframed}
\usepackage[shortlabels]{enumitem}
\usepackage{indentfirst}
\usepackage{hyperref}
\usepackage{mhchem}
\renewcommand{\thesubsection}{\thesection.\alph{subsection}}

\newenvironment{problem}[2][Problem]
    { \begin{mdframed}[backgroundcolor=gray!20] \textbf{#1 #2} \\}
    {  \end{mdframed}}

% Define solution environment
\newenvironment{solution}
    {\textit{Solution:}}
    {}

\renewcommand{\qed}{\quad\qedsymbol}
%%%%%%%%%%%%%%%%%%%%%%%%%%%%%%%%%%%%%%%%%%%%%%%%%%%%%%%%%%%%%%%%%%%%%%%%%%%%%%%%%%%%%%%%%%%%%%%%%%%%%%%%%%%%%%%%%%%%%%%%%%%%%%%%%%%%%%%%
\begin{document}\noindent
\textbf{Aakash Ghosh\hfill Machine Learning\\
19MS129\hfill LAB-5\\}
\begin{problem}{1}
    Write a code for that carries out leave one out cross validation
    and displays the classification rate. You should use the dataset
    as input and the output will be the average classification rate.
\end{problem}
\begin{solution}
For classes $\{\omega_i\}_{1\leq i\leq n}$, we assign observation $x$ to class $\omega_i$ if $P(i|x)\geq P(j|x)$ for all $j\ne i$. We assume dataset of each class follows a normal distribution. Let the PDF for class $i$ be given by $p_i$. Then:
$$p(i|x)=\frac{p(x|i)p(i)}{p(x)}=p_i(x)\times\text{No. of obs from class $i$}\times\frac{1}{p(x)\times\text{Total No. of obs}}$$  
\end{solution} 
As $\left(\frac{1}{p(x)\times\text{Total No. of obs}}\right)$ is constant for a data set, we can consider $g_i(x)=p_i(x)\times\text{No. of obs from class $i$}$ to be the discriminant function. For IRIS data we get a classification rate of $97.33\%$ and for wine data we get a rate of $99.44\%$. The relevant code is given below
\lstinputlisting{feb21_19ms129.m}

\noindent\rule{7in}{2.8pt}

\end{document}
 