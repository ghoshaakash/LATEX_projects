\documentclass{article}
\usepackage{amsmath}
\usepackage{amssymb,amsthm} 
\usepackage{graphicx}
\usepackage{tikz}
\usetikzlibrary{arrows}
\usepackage{verbatim}
\usepackage{float}
\usepackage{tikz}
    \usetikzlibrary{shapes,arrows}
    \usetikzlibrary{arrows,calc,positioning}

    \tikzset{
        block/.style = {draw, rectangle,
            minimum height=1cm,
            minimum width=1.5cm},
        input/.style = {coordinate,node distance=1cm},
        output/.style = {coordinate,node distance=4cm},
        arrow/.style={draw, -latex,node distance=2cm},
        pinstyle/.style = {pin edge={latex-, black,node distance=2cm}},
        sum/.style = {draw, circle, node distance=1cm},
    }
\usepackage{xcolor}
\usepackage{mdframed}
\usepackage{hyperref}
\usepackage{mhchem}
\renewcommand{\thesubsection}{\thesection.\alph{subsection}}

\newenvironment{problem}[2][Problem]
    { \begin{mdframed}[backgroundcolor=gray!20] \textbf{#1 #2} \\}
    {  \end{mdframed}}

% Define solution environment
\newenvironment{solution}
    {\textit{Solution:}}
    {}

\renewcommand{\qed}{\quad\qedsymbol}

\title{Solutions for Econometrics Class Test 1 }
\author{Aakash Ghosh }
\date{November 2022}

\usepackage{amsmath}


\newtheorem{defn}{Definition}
\newtheorem{theorem}{Theorem}
\newtheorem{corollary}{Corollary}[theorem]
\newtheorem{lemma}[theorem]{Lemma}


\usepackage {tfrupee}%%Use rupee by \rupee

\usepackage{listings}
\usepackage{color}

\definecolor{dkgreen}{rgb}{0,0.6,0}
\definecolor{gray}{rgb}{0.5,0.5,0.5}
\definecolor{mauve}{rgb}{0.58,0,0.82}

\lstset{frame=tb,
  language=R,
  aboveskip=3mm,
  belowskip=3mm,
  showstringspaces=false,
  columns=flexible,
  basicstyle={\small\ttfamily},
  numbers=none,
  numberstyle=\tiny\color{gray},
  keywordstyle=\color[HTML]{172774},
  commentstyle=\color[HTML]{EEEEEE},
  stringstyle=\color[HTML]{E94560},
  breaklines=true,
  breakatwhitespace=true,
  tabsize=3
}%%USe between begin and end lstlisting
\usepackage{ stmaryrd }


\makeatletter


\begin{document}
\lstset{language=R}

\large\textbf{Aakash Ghosh} \hfill \textbf{}   \\
Email: ag19ms129@iiserkol.ac.in  \hfill ID: 19MS129 \\
\normalsize Course: HU4102 \hfill Term: Autmn 2022\\
\noindent\rule{7in}{2.8pt}
\section{Data Preparation}

\section{Solutions}
\begin{problem}{1}
Find the gender ratio in 0-15 yrs old for Haryana and Kerala.
\end{problem}
\begin{solution}

\end{solution} 

\begin{problem}{2}
Compare the percapita consumption expenditure of UP and TN.   
\end{problem}
\begin{solution}
    
\end{solution}


\begin{problem}{3}
Prepare an educational tranition matrix education level of mothers and their sons.
\end{problem}
\begin{solution}
    
\end{solution}


\begin{problem}{4}
What percent of sons are employed in agriculture given their fathers work in agriculture.
\end{problem}
\begin{solution}
    
\end{solution}
\end{document}
