%!TEX program = lualatex
\documentclass{beamer}
\usetheme{metropolis} 

% Presento style file
\usepackage{config/presento}
\usepackage{amsmath}
\usepackage{amssymb}
\usepackage{tcolorbox}

\usepackage{physics}
\usepackage{amsmath}
\usepackage{tikz}
\usepackage{mathdots}
\usepackage{yhmath}
\usepackage{cancel}
\usepackage{color}
\usepackage{siunitx}
\usepackage{array}
\usepackage{multirow}
\usepackage{amssymb}
\usepackage{gensymb}
\usepackage{tabularx}
\usepackage{extarrows}
\usepackage{booktabs}
\usetikzlibrary{fadings}
\usetikzlibrary{patterns}
\usetikzlibrary{shadows.blur}
\usetikzlibrary{shapes}
\usepackage{pifont}

% custom command and packages
\input{config/custom-command}
 
% Information
\title{Proposal on Machine Learning Classification of dermatological diseases}
\subtitle{}
\author{Aakash Ghosh}
\institute{19MS129}
\date{$1^{st}$ April,2023}

\begin{document}

% Title page
\begin{frame}[plain]
\maketitle
\end{frame}

% sections in the presentation
\begin{frame}{Timeline}    
    We are going to look at:
    \begin{enumerate}
        \item Current state
        \item Problem Statement
        \item Advantages, Issues and Prospective Remedies
        \item A Naive Roadmap
        \item Questions
    \end{enumerate}
\end{frame}




\framecard[colorgreen]{{\color{white}\hugetext{Current state of affairs}}}
\begin{frame}{Current state of affairs}
    We look at the current state of affairs regarding digital health records in India\pause
    \begin{enumerate}
         \item Post diagnosis treatment and prescription is done on a handwritten basis in India.\pause
         \item There is no standardised digital system for storing data and using it for statistical analysis in India.\pause
         \item National Health Stack and National Digital Health Mission (NDHM) are initiatives to create a digital health ecosystem in India.\pause
         \item The benefits of a fully digital healthcare system include improved patient outcomes, reduced medical errors, and better management of chronic diseases.
    \end{enumerate}
\end{frame}
\framecard[colorgreen]{{\color{white}\hugetext{Problem statement}}}
\begin{frame}{Problem statement}
    Ideally we would like to have a system:
    \begin{enumerate}
        \item Which stores a patients' health Information\pause
        \item Gives predictive diagnosis based on history and symptoms\pause
        \item Updates itself based on new information.
    \end{enumerate}
\end{frame}


\framecard[colorgreen]{{\color{white}\hugetext{Advantages and Disadvantages}}}

\begin{frame}{Advantages of ML based Diagnosis systems}
    \begin{enumerate}
        \item Improved Accuracy: Machine learning models can analyze vast amounts of data and provide accurate diagnoses based on patterns and trends that may be missed by human doctors.\pause
        \item Early Detection: By analyzing patient data in real-time, machine learning models can detect potential health issues at an early stage, enabling prompt and effective treatment.\pause
        \item Personalized Treatment: Machine learning models can analyze patient data and provide personalized treatment plans, taking into account individual factors such as age, gender, and medical history.\pause
        \item Increased Efficiency: With the use of machine learning models, healthcare providers can make more efficient use of their time, reducing wait times for patients.
        \end{enumerate}
    
\end{frame}
\begin{frame}{Disadvantages of ML based Diagnosis systems}
    \textbf{Presence of selection bias in training data}

Medical diagnoses involve highly personal information, which can result in selection bias in the data used to train our models. To address this issue, we suggest two remedies:

\begin{enumerate}
\item \textbf{Short-term remedy:} Use matching techniques to minimize selection bias.\pause
\item \textbf{Long-term remedy:} Obtain anonymous data at the time of diagnosis to eliminate initial bias.
\end{enumerate}
\end{frame}
\begin{frame}{Disadvantages of ML based Diagnosis systems}
    \textbf{Anchoring bias}

Psychological constraints such as anchoring bias can affect diagnostic accuracy. To mitigate this issue, we suggest providing a small list of highly likely diagnoses in a randomized order. This approach can speed up the diagnostic process for medical professionals while avoiding the negative effects of anchoring.\pause

One can argue that diagnosis after a machine prediction is made is equivalent to getting a second opinion: therefore there is as much anchoring as there would be in most general cases.
\end{frame}


\framecard[colorgreen]{{\color{white}\hugetext{Naive roadmap}}}

\begin{frame}{Naive Roadmap}

    \begin{enumerate}
        \item Using a mixed mode: First use decision trees to segregate influencing factors such as demography, gender(?) and age, then using a model based on the leaf reached. \pause
        \item Method of proper outlier analysis needed.\pause
        \item Models can be parametric in low depth and unparametric in high depth. On can also go for efficient matching techniques(similar to KNN).
    \end{enumerate}
\end{frame}


\begin{frame}{Topics on which we need more Information}
    \begin{enumerate}
        \item In cases where the preliminary symptoms don't give conclusive results, how are further diagnosis steps taken?
        \item Are symptoms dependent on demography/skin colour? How conclusively can demography be determined? How are mixed-racial patients treated?
        \item How to determine cost function? Is it disease specific?
    \end{enumerate}

\end{frame}

\end{document}